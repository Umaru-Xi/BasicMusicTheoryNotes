\documentclass[a4paper,20pt]{article}
\usepackage{geometry}
\usepackage{savesym}
\usepackage{amsmath}
\usepackage{esint}
\usepackage{mathrsfs}
\usepackage{amssymb}

\usepackage[UTF8]{ctex}
\savesymbol{breve}
\usepackage{musixtex}
\restoresymbol{breve}{breve}

\input{musixlit}


\geometry{left=1.5cm, right=1.5cm, top=1.5cm, bottom=1.5cm}
\setlength{\lineskip}{0.75em}
\setlength{\parskip}{0.75em}

\begin{document}
\begin{center} 
 \Large \textbf{音乐理论基础}\par
 \textbf{十六.转调及交替调式}
\end{center}

\large 
\begin{center}
 \textbf{1.转调}\\
\end{center}

[\textbf{转调}] 在音乐作品中, 由一个调进行到另一个调被称为转调.\par

[\textbf{转调的意义}]\par
\qquad 1. 当音乐离开主调(稳定的调)进入新调时, 稳定性被破坏却依然要求继续进行, 这样就形成了主调与新调间稳定与不稳定的对比;\par
\qquad 2. 转调可以改变乐曲的色彩.\par

[\textbf{转调的结构关系}] 转调在结构上分为对置式和过渡式.\par
\qquad 1. 对置式: 不加任何过渡的改变调性;\par
\qquad 2. 过渡式.\par

[\textbf{转调的时间关系}] 转调在时间上分为转调和临时转调.\par
\qquad 1. 转调: 代替主调的新调被巩固, 并用它来结束音乐;\par
\qquad 2. 临时转调: 短暂且过渡性质的, 不发生在段落的结束处, 只发生在段落的中间.\par

[\textbf{转调的远近关系}] 转调在远近关系上分为同音列转调, 同主音转调和不同主音不同音列的转调.\par
\qquad 1. 同音列转调: 只转移主音, 不改变音列. 如: 同宫系统转调和关系大小调转调;\par
\qquad 2. 同主音转调: 只改变音列, 不改变主音. 如: 同主音大小调;\par
\qquad 3. 不同主音不同音列的转调.\par
\qquad 注意: 同主音转调和不同主音不同音列的转调都可以是不同宫系统转调.\par

[\textbf{不同手法的转调}] 根据转调手法, 有模进转调, 过渡转调和突然转调等.\par

\begin{center}
 \textbf{2.调的关系}\\
\end{center}

[\textbf{调的远近关系判定}] 两调之间的共同音和共同和弦的多少.\par
\qquad 推论: 相差一个调号的宫系统和大小调的关系是较近的.\par

[\textbf{近关系调}] 调号相同或相差一个升降号的各调.\par
\qquad C宫调式的近关系调: \par
\qquad \qquad 1. 同宫系统各调: D商调式, E角调式, G徵调式, A羽调式;\par
\qquad \qquad 2. 上方无度宫系统各调: G宫调式, A商调式, B角调式, D徵调式, E羽调式;\par
\qquad \qquad 3. 下方五度宫系统各调: F宫调式, G商调式, A角调式, C徵调式, D羽调式;\par
\clearpage
\qquad C大调的近关系调:\par
\qquad \qquad 1. 本调的关系调: a小调;\par
\qquad \qquad 2. 上方五度的大调: G大调, 及其关系小调: e小调;\par
\qquad \qquad 3. 下方五度的大调: F大调, 及其关系小调: d小调.\par
\qquad a小调的近关系调:\par
\qquad \qquad 1. 本调的关系大调: C大调;\par
\qquad \qquad 2. 上方五度的小调: e小调, 及其关系大调: G大调;\par
\qquad \qquad 3. 下方五度的小调: d小调, 及其关系大调: F大调.\par
\qquad 注意: 部分教材中, 大调的近关系调还包括下方五度的小调, 小调的近关系调还包括上方五度的大调.\par

[\textbf{远关系调}] 除近关系调外的调被认为关系较远.\par

\begin{center}
 \textbf{3.交替调式}\\
\end{center}

[\textbf{交替调式}] 在调关系的扩大和发展中, 为了丰富调式的表现力, 将一些关系较近的调结合在一起, 成为一个新的调式体系. 这种新的调式体系被称为交替调式.\par

[\textbf{同音列交替}] 构成交替的两个调的音列是相同的, 不同的只有调式的主音. 且调式不必相同.\par
\qquad 五声调式体系最常见的同音列交替是具有强烈支持力的四度和五度关系, 如徵宫交替, 徵商交替, 商羽交替等;\par
\qquad 大小调体系中最常见的同音列交替是关系大小调交替, 也被称为平行交替.\par

[\textbf{同主音交替}] 有两个同主音的调交替构成, 不同的只是调式的音列.\par
\qquad 在大小调体系中同主音交替即同主音大小调.\par

[\textbf{不同主音不同音列交替}] 构成交替的两种调式的主音和音列都不同, 但调式可能相同也可能不同.\par

\end{document}
