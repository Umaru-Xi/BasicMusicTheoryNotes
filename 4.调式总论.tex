\documentclass[a4paper,20pt]{article}
\usepackage{geometry}
\usepackage{savesym}
\usepackage{amsmath}
\usepackage{esint}
\usepackage{mathrsfs}
\usepackage{amssymb}

\usepackage[UTF8]{ctex}
\savesymbol{breve}
\usepackage{musixtex}
\restoresymbol{breve}{breve}

\input{musixlit}


\geometry{left=1.5cm, right=1.5cm, top=1.5cm, bottom=1.5cm}
\setlength{\lineskip}{0.75em}
\setlength{\parskip}{0.75em}

\begin{document}
\begin{center} 
 \Large \textbf{音乐理论基础}\par
 \textbf{四.调式总论}
\end{center}

\large 
\begin{center}
 \textbf{1.调式}\\
\end{center}

[\textbf{调式}] 按照一定关系连接在一起的许多音(一般不超过7个), 组成一个以一个音为中心(主音)的体系, 称之为调式.\par

[\textbf{音阶}] 调式中的音按照上行或下行的高低次序, 由主音到主音排列起来, 称之为音阶.\par
\qquad 注意: 音阶与音列不同. 音阶在一定程度上能表现调式的规律, 音列只能是构成调式的要素.\par

[\textbf{调}] 调是调式的音高位置. 调的名称由主音标记和调式标记两部分组成.\par

[\textbf{调性}] 调本声所具有的性质.\par

[\textbf{调式音级}] 调式体系中的各音.\par
\qquad 与乐音体系中的音级区别: 乐音体系中的音级只是音高的物理关系; 调式音级按照一定的关系结合在一起, 每一个音都具有特别的调式意义.\par

[\textbf{稳定音与不稳定音}] 调式体系中起支撑作用并给人以稳定感的音称为稳定音, 反之.\par
\qquad 注意: 音的稳定与不稳定是相对的, 不是绝对的. 在不同调式体系中可能发生变化, 在同一调式中由于和声处理的不同也可能变化.\par

[\textbf{主音}] 稳定音中最稳定的具有中心作用的音.\par

[\textbf{倾向与解决}] 不稳定音具有进行到稳定音的特性, 这种特性称为倾向. 不稳定音根据其倾向进行到稳定音称为解决.\par

[\textbf{和弦的稳定}] 在多声部音乐中, 各声部纵向的关系形成和弦, 起稳定作用的将是整个和弦.\par
\qquad 通常以主音上的大小三和弦为稳定和弦(有例外), 其他一切和弦都与主音和弦形成对比作为不稳定和弦.\par

\end{document}
