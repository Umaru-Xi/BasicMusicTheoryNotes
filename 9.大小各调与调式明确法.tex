\documentclass[a4paper,20pt]{article}
\usepackage{geometry}
\usepackage{savesym}
\usepackage{amsmath}
\usepackage{esint}
\usepackage{mathrsfs}
\usepackage{amssymb}

\usepackage[UTF8]{ctex}
\savesymbol{breve}
\usepackage{musixtex}
\restoresymbol{breve}{breve}

\input{musixlit}


\geometry{left=1.5cm, right=1.5cm, top=1.5cm, bottom=1.5cm}
\setlength{\lineskip}{0.75em}
\setlength{\parskip}{0.75em}

\begin{document}
\begin{center} 
 \Large \textbf{音乐理论基础}\par
 \textbf{九.大小各调与调式明确法}
\end{center}

\large 
\begin{center}
 \textbf{1.大小各调}\\
\end{center}

[\textbf{关系大小调原理}] 在自然形式中, 音的组织相同因而调号也相同的大小调被称为关系大小调, 也称为平行调.\par
\qquad 如: C大调的关系小调(a小调)好象是以大调的第VI级(主音下方小三度)作为主音构成的, 反之a小调的关系大调(C大调)好象是以小调的第III级(主音上方小三度)作为主音构成的.\par
\startextract
\Notes \wh{cdefghij}\en\bar
\Notes \wh{abcdefgh}\en
\zendextract

[\textbf{大调各调}] 大调可以在任何音级上构成, 除了没有升降号的C大调外, 还有包含升降音的各7个大调.\par
\qquad 下谱各小节依次示: C大调, G大调, $^\# F$大调, $^b B$大调. 包含调号, 稳定音级和音列.\par
\qquad 和声小调和旋律小调同理构成.\par
\startextract
\Notes \zh{ce}\wh g \wh{cdefghij}\en\bar
\Notes \sh{m} \zh{gi}\wh k \wh{ghijklmn}\en
\zendextract
\startextract \barno=3
\Notes \sh m \sh j \sh n \sh k \sh h \sh l \zh{fh}\wh j \wh{fghijklm}\en\bar
\Notes \fl i \fl l \zh{ik}\wh m \wh{ijklmnop}\en
\zendextract

[\textbf{小调各调}] 同大调各调的原理可构成小调各调.\par
\qquad 下谱各小节依次示: a小调, e小调, $^\# f$小调, d小调. 包含调号, 稳定音级和音列.\par
\qquad 和声小调和旋律小调同理构成.\par
\startextract
\Notes \zh{hj}\wh l \wh{hijklmno}\en\bar
\Notes \sh l \zh{eg}\wh i \wh{efghijkl}\en
\zendextract
\startextract \barno=3
\Notes \sh l \sh j \sh n \zh{fh}\wh j \wh{fghijklm}\en\bar
\Notes \fl i \zh{df}\wh h \wh{defghijk} \en
\zendextract

\clearpage

\begin{center}
 \textbf{2.大小调的比较}\\
\end{center}

[\textbf{同主音大小调}] 同一音级为主音的大小调叫做同主音大小调, 也称为同名调. \par
\qquad 例如: C大调和c小调.\par

[\textbf{自然形式的同主音大小调}] 同主音大小调的自然形式有三个音级不同, 第III级, 第VI级和第VII级. 在小调中, 这三个音级比大调低一变化半音, 因此同主音大小调的调号总是相差三个升降号.\par
\startextract
\Notes \zh{ce}\wh g \wh{cdefghij} \en\bar
\Notes \fl i \fl l \fl h \wh{cd} \fl e\wh{efg} \fl h\wh h \fl i\wh i \wh j \en
\zendextract

[\textbf{和声形式的同主音大小调}] 和声形式的同主音大小调, 只有一个音级不同, 即第III级. 它们的共同特征是第VI级与第VII级间的增二度. \par
\startextract
\Notes \wh{cdefg} \fl h \wh{hij}\en\bar
\Notes \fl i \fl l \fl h \wh{cdefgh} \na i\wh i \wh j\en
\zendextract

[\textbf{旋律形式的同主音大小调}] 旋律形式的同主音大小调, 除第III级有差别外, 大调式的旋律形式和小调式的自然形式相似, 而小调式的旋律形式则和大调式的自然形式相似. \par

[\textbf{大小调的特征}] 大小调中唯一可靠的特征是主音上的三度, 与主音成大三度则为大调, 成小三度则为小调. \par
\qquad 第VI级和第VII级虽然对大小调性也有影响, 但他们变化是多端的, 其他音级在各种形式的大小调式中则都是一样的, 因此对调性没有影响.\par

[\textbf{各类大调的表现特性}] \par
\qquad 1. 自然大调: 自然大调是三种大调中的基本形式, 色彩明朗, 应用最为广泛.\par
\qquad 2. 和声大调: 和声大调产生较晚, 应用不像自然大调那么普遍, 由于降VI级的关系, 这一调式的色彩比较暗淡, 柔和, 有点像小调的感觉.\par
\qquad 3. 旋律大调: 旋律大调产生最晚, 在应用上受到很大的限制, 因此一般较少应用. \par

[\textbf{各类小调的表现特性}] 三种形式的小调的应用都比较普遍, 从总的来说, 小调的色彩一般都比较暗淡. \par
\qquad 和声小调: 和声小调由于升高第VII级而造成导音向主音的倾向异常尖锐, 因而有近似大调的特点. \par

\clearpage

\begin{center}
 \textbf{3.特种自然大小调}\\
\end{center}

[\textbf{混合利底亚调式}] 由降低自然大调第VII级而造成的主音上的小七度(混合利底亚七度). \par
\startextract
\Notes \wh{cdefgh} \fl i \wh{ij}\en
\zendextract

[\textbf{利底亚调式}] 由升高自然大调第IV级而造成主音上的增四度(利底亚四度). \par
\startextract
\Notes \wh{cde} \sh f \wh{fghij}\en
\zendextract

[\textbf{多利亚调式}] 由升高自然小调第VI级(多利亚六度)而构成.\par
\startextract
\Notes \wh{abcde} \sh f \wh{fgh}\en
\zendextract

[\textbf{弗里吉几调式}] 由升高自然小调第II级(弗里几亚二度)而构成.\par
\startextract
\Notes \wh a \fl b \wh{bcdefgh}\en
\zendextract

[\textbf{混合自然大小调的特性}]\par
\qquad 混合利底亚调式和利底亚调式: 由于主音上的三和弦是大三和弦, 因此具有大调的特性.\par
\qquad 弗里几亚调式和多利亚调式: 由于主音上的三和弦是小三和弦, 因此具有小调的特性.\par

\begin{center}
 \textbf{4.作品中调的明确法}\\
\end{center}

[\textbf{主观明确法}] 明确作品的调, 首先是依靠听觉, 它往往能够清楚地感觉出乐曲的基本调式和旋律进行的特点(以五声调式为基础的各种调式或大小调). \par

[\textbf{客观明确法}] \par
\qquad 1. 利用旋律中音的组织及表示音组织的调号, 如雅乐音阶, 燕乐音阶, 和声调式, 旋律调式中的临时记号;\par
\qquad 2. 音乐的结束音, 一般情况下, 乐曲的结束音总是主音;\par
\qquad 3. 音乐的结束和弦, 乐曲的结束总是主和弦, 而主和弦的低音总是主音.\par

[\textbf{相同音阶调式的明确}] 首先应从旋律风格上去区分大小调体系与五声音阶为基础的调式体系.\par
\qquad 对于五声音阶为基础的调式体系, 应先找出五声音阶的基础, 再根据基础来确定调式.\par
\qquad 在同时具有大小调和五声音阶调式特征的情况下, 应根据其主要方面确定调式.\par

\end{document}
