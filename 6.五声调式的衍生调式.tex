\documentclass[a4paper,20pt]{article}
\usepackage{geometry}
\usepackage{savesym}
\usepackage{amsmath}
\usepackage{esint}
\usepackage{mathrsfs}
\usepackage{amssymb}

\usepackage[UTF8]{ctex}
\savesymbol{breve}
\usepackage{musixtex}
\restoresymbol{breve}{breve}

\input{musixlit}


\geometry{left=1.5cm, right=1.5cm, top=1.5cm, bottom=1.5cm}
\setlength{\lineskip}{0.75em}
\setlength{\parskip}{0.75em}

\begin{document}
\begin{center} 
 \Large \textbf{音乐理论基础}\par
 \textbf{六.五声调式的衍生调式}
\end{center}

\large 
\begin{center}
 \textbf{1.六声和七声调式}\\
\end{center}

[\textbf{六声调式的音}] 在五声调式基础上加入清角音(角音上方小二度), 或变宫音(宫音下方小二度).\par
\qquad 下谱按清角, 宫, 徵, 商, 羽, 角, 变宫的顺序排列.\par
\startextract
\Notes \wh f\wh j\wh g\wh k\wh h\wh l\wh i\en
\zendextract

[\textbf{六声调式}] 依照C宫调式, G徵调式, D商调式, A羽调式和E角调式的顺序, 每组分为加清角或加变宫两种.\par
\qquad 虽然加清角的宫调式和加变宫的徵调式在音阶的音程上是完全一样, 但仍然是两种完全不同的调式, 因为这两种调式的五声基础不同. 其他调式同样. \par
\startextract
\Notes \wh c\wh d\wh e\wh f\wh g\wh h\wh j\en\bar
\Notes \wh c\wh d\wh e\wh g\wh h\wh i\wh j\en
\zendextract
\startextract\barno=3
\Notes \wh g\wh h\wh j\wh k\wh l\wh m\wh n\en\bar
\Notes \wh g\wh h\wh i\wh j\wh k\wh l\wh n\en
\zendextract
\startextract\barno=5
\Notes \wh d\wh e\wh f\wh g\wh h\wh j\wh k\en\bar
\Notes \wh d\wh e\wh g\wh h\wh i\wh j\wh k\en
\zendextract
\startextract\barno=7
\Notes \wh h\wh j\wh k\wh l\wh m\wh n\wh o\en\bar
\Notes \wh h\wh i\wh j\wh k\wh l\wh n\wh o\en
\zendextract
\startextract\barno=9
\Notes \wh e\wh f\wh g\wh h\wh j\wh k\wh l\en\bar
\Notes \wh e\wh g\wh h\wh i\wh j\wh k\wh l\en
\zendextract

[\textbf{正音和偏音}] 正音: 宫徵商羽角; 偏音: 清角和变宫.\par

[\textbf{七声调式的音}] 在五声调式的小三度音程中间加入不同的偏音构成, 加入的音分三类.\par
\qquad 主要使用的是清乐音阶, 雅乐音阶和燕乐音阶只在个别地区和个别作品中使用.\par
\qquad 1. 雅乐音阶: 加入变徵(徵音下方小二度)和变宫;\par
\qquad 2. 清乐音阶: 加入清角和变宫两音;\par
\qquad 3. 燕乐音阶: 加入清角和闰(宫音下方大二度).\par

\clearpage

[\textbf{七声调式}] 依照C宫调式, G徵调式, D商调式, A羽调式和E角调式的顺序. 每组分为雅乐音阶, 清乐音阶和燕乐音阶三种.\par
\qquad 各调式中的清乐音阶和燕乐音阶的音程结构均与后一调式中的雅乐音阶和清乐音阶相同, 但因无声基础不同而是完全不同的调式音阶.\par
\startextract
\Notes \wh c\wh d\wh e\sh{f}\wh f\wh g\wh h\wh i\wh j\en\bar
\Notes \wh c\wh d\wh e\wh f\wh g\wh h\wh i\wh j\en\bar
\Notes \wh c\wh d\wh e\wh f\wh g\wh h\fl{i}\wh i\wh j\en
\zendextract
\startextract\barno=4
\Notes \wh g\wh h\wh i\wh j\wh k\wh l\sh{m}\wh m\wh n\en\bar
\Notes \wh g\wh h\wh i\wh j\wh k\wh l\wh m\wh n\en\bar
\Notes \wh g\wh h\fl{i}\wh i\wh j\wh k\wh l\wh m\wh n\en
\zendextract
\startextract\barno=7
\Notes \wh d\wh e\sh{f}\wh f\wh g\wh h\wh i\wh j\wh k\en\bar
\Notes \wh d\wh e\wh f\wh g\wh h\wh i\wh j\wh k\en\bar
\Notes \wh d\wh e\wh f\wh g\wh h\fl{i}\wh i\wh j\wh k\en
\zendextract
\startextract\barno=10
\Notes \wh h\wh i\wh j\wh k\wh l\sh{m}\wh m\wh n\wh o\en\bar
\Notes \wh h\wh i\wh j\wh k\wh l\wh m\wh n\wh o\en\bar
\Notes \wh h\fl{i}\wh i\wh j\wh k\wh l\wh m\wh n\wh o\en
\zendextract
\startextract\barno=13
\Notes \wh e\sh{f}\wh f\wh g\wh h\wh i\wh j\wh k\wh l\en\bar
\Notes \wh e\wh f\wh g\wh h\wh i\wh j\wh k\wh l\en\bar
\Notes \wh e\wh f\wh g\wh h\fl{i}\wh i\wh j\wh k\wh l\en
\zendextract

[\textbf{七声调式的特点}] 和五声调式相比增加了半音和三整音的音程关系.\par

\begin{center}
 \textbf{2.同宫系统 \& 五声衍生调式的调式音级}\\
\end{center}

[\textbf{同宫系统}] 以同一音级为宫的调式称为同宫系统各调.\par

[\textbf{同宫系统各调}] 同宫系统各调所用的音是共同的, 区别只在于采用了不同的主音. 音列中任何一个音级(基本或变化)都可以作为宫音而建立一个同宫系统. \par

[\textbf{五声调式的调式音级}] 五声调式的调式音级(宫徵商羽角)只代表调式中各音级间的固定音高关系, 不表明音的绝对高度. 这些音级名称在同宫系统各调中是永远固定不变的. 在六声调式和七声调式中, 所加的调式音级(清角, 变徵, 闰, 变宫等偏音)同样.\par

[\textbf{调式音级的号数}] 用罗马数字标记的调式音级, 依照上行依次是: I, II, III, IV, V, VI, VII等. 在本调中, 不论调式音级如何排列, 其调式音级的号数任然固定不变. 为了音级号数标记的简便统一, 五声调式和六声调式中都以同名的七声调式音级的号数来标记. 但不意味着它们是不完整的调式.\par
\qquad 下谱按照C宫调清乐音阶I到VII的顺序排列.\par
\startextract
\Notes \wh c\wh d\wh e\wh f\wh g\wh h\wh i\en
\zendextract

\clearpage

[\textbf{调号}] 为了建立新的同宫系统而需要使用的变音记号, 是按照一定的次序和位置记在谱号后面的, 称之为调号.\par
\qquad 调号只用同类的变音记号, 只用升记号或只用降记号. 因此可以将调分为包含升号的调和包含降号的调.\par
\qquad 雅乐音阶的变徵和燕乐音阶的闰所使用的变音记号一般用临时记号标记, 不记在调号中.\par

[\textbf{包含升号的同宫系统}] 从没有升降号的C宫系统开始, 按照纯五度音程关系向上依次有G, D, A, E, B为宫音的同宫系统(G宫系统, D宫系统...), 之后还可依次产生$^\#F$, $^\#C$, $^\#G$...等.\par
\qquad 下谱各小节依次是G, D, A, E, B, $^\#F$, $^\#C$的同宫系统调号.\par
\startextract
\Notes \sh{m}\en\bar
\Notes \sh{m} \sh{j}\en\bar
\Notes \sh{m} \sh{j} \sh{n}\en\bar
\Notes \sh{m} \sh{j} \sh{n} \sh{k}\en\bar
\Notes \sh{m} \sh{j} \sh{n} \sh{k} \sh{h}\en\bar
\Notes \sh{m} \sh{j} \sh{n} \sh{k} \sh{h} \sh{l}\en\bar
\Notes \sh{m} \sh{j} \sh{n} \sh{k} \sh{h} \sh{l} \sh{i}\en
\zendextract

[\textbf{包含降号的同宫系统}] 下谱各小节依次是F, $^bB$, $^bE$, $^bA$, $^bD$, $^bG$, $^bC$为宫音的同宫系统.\par
\startextract
\Notes \fl{i}\en\bar
\Notes \fl{i} \fl{l}\en\bar
\Notes \fl{i} \fl{l} \fl{h}\en\bar
\Notes \fl{i} \fl{l} \fl{h} \fl{k}\en\bar
\Notes \fl{i} \fl{l} \fl{h} \fl{k} \fl{g}\en\bar
\Notes \fl{i} \fl{l} \fl{h} \fl{k} \fl{g} \fl{j}\en\bar
\Notes \fl{i} \fl{l} \fl{h} \fl{k} \fl{g} \fl{j} \fl{f}\en
\zendextract
\qquad 备注: 调号中的升降号数目一般不超过七个, 但在乐曲的部分片断中可能出现多于七个的升降号的调.\par

[\textbf{五度循环}] 各个宫系统(或调)按照纯五度关系排列起来称为五度循环.\par
\qquad 五度循环理论上可以无穷继续下去, 但由于等音关系, 有条件使五度循环在$^bD$与$^\#C$, $^bG$与$^\#F$, $^bC$与$B$三对宫系统中开始合成一个圆圈.\par
\qquad 五度循环向上进行: 按照五度向上, 使升号增加或降号减少;\par
\qquad 五度循环向下进行: 按照五度向下, 使升号减少或降号增加.\par
\qquad 备注: 五度循环进行中, 升号和降号增加的次序是相反的.\par

[\textbf{等音宫系统}] 两个宫系统(或调)中所有音级都是等音, 而且具有同样的调式意义(I=I, II=II等), 这样的宫系统(或调)称为等音宫系统(或等音调).\par
\qquad 在等音宫系统中总是一个属于升号调, 另一个属于降号调.\par

[\textbf{等音调升降号的同宫系统}] 在前述七个升降号系统中, 有三对等音调.\par
\qquad 1. B宫系统($5^\#$) 与 $^bC$同宫系统($7^b$);\par
\qquad 2. $^\#F$宫系统($6^\#$) 与 $^bG$同宫系统($6^b$);\par
\qquad 3. $^\#C$宫系统($7^\#$) 与 $^bD$同宫系统($5^b$).\par

[\textbf{偏音在旋律中的作用}] 偏音在旋律中多半具有装饰性, 并以经过音和辅助音的形式出现. 但其在音乐表现中依然重要.\par

\clearpage

[\textbf{以五声音阶衍生调式的特性}] \par
\qquad 1. 五声调式由于缺少半音和三整音的音程关系, 因此格调比较平和, 音的倾向性不那么尖锐, 常见于民歌中;\par
\qquad 2. 一般来说, 宫调式和徵调式的色彩比较明亮, 羽调式和角调式的色彩较暗淡, 商调式介于两者之间;\par
\qquad 3. 六声调式和七声调式由于是五声调式衍生, 总的格调与五声调式近似. 但由于偏音的加入产生了新的音程, 使得其多用于戏曲音乐中;\par
\qquad 4. 五声调式是这一调式的基础, 对研究五声调式体系中旋律音调进行的特点有重要意义; 当为这一调式配置和声时, 则多使用七声调式.\par

\end{document}
