\documentclass[a4paper,20pt]{article}
\usepackage{geometry}
\usepackage{savesym}
\usepackage{amsmath}
\usepackage{esint}
\usepackage{mathrsfs}
\usepackage{amssymb}

\usepackage[UTF8]{ctex}
\savesymbol{breve}
\usepackage{musixtex}
\restoresymbol{breve}{breve}

\input{musixlit}


\geometry{left=1.5cm, right=1.5cm, top=1.5cm, bottom=1.5cm}
\setlength{\lineskip}{0.75em}
\setlength{\parskip}{0.75em}

\begin{document}
\begin{center} 
 \Large \textbf{音乐理论基础}\par
 \textbf{八.大小调式基础}
\end{center}

\large 
\begin{center}
 \textbf{1.大小调式}\\
\end{center}

[\textbf{大调式}] 大调式由七个音组成, 其中稳定音合起来组成一个大三和弦.\par
\qquad 大调式的主音和其上方第三音为大三度, 这个音程最能说明大调式的色彩.\par

[\textbf{大调式的形式}] 大调式有三种形式.\par
\qquad 1. 自然大调: 大调式的基本形式, 从主音开始逐次升高大二度, 大二度, 小二度, 大二度, 大二度, 大二度, 小二度. 图示第一小节为C自然大调;\par
\qquad 2. 和声大调: 是降低自然大调中的VI级而构成的, 其明显的特点是第VI级与第VII级间的增二度音程. 图示第二小节为C和声大调;\par
\qquad 3. 旋律大调: 将主音下行级进时的自然大调降低第VII级, 第VI级而形成. 图示第三小节为\par
\startextract
\Notes \wh{cdefghij} \en\bar
\Notes \wh{cdef} \fl g\wh {ghij} \en\bar
\Notes \wh j \fl i\wh i \fl h\wh h \wh{gfedc} \en
\zendextract

[\textbf{小调式}] 小调式由七个音组成, 其中稳定音合起来成为一个小三和弦.\par
\qquad 小调式的主音和其上方第三音为小三度, 因为这个音程最能说明小调式的色彩.\par

[\textbf{小调式的形式}] 小调式有三种形式.\par
\qquad 1. 自然小调: 小调式的基本形式, 从主音开始逐次升高大二度, 小二度, 大二度, 大二度, 小二度, 大二度, 大二度. 图示第一小节为a自然小调.\par
\qquad 2. 和声小调: 是升高自然小调第VII级而成. 因此和声小调的第VII级音向主音的倾向和大调一样尖锐. 为了区别和声小调与自然小调中第VII级音的不同, 自然小调中的第VII级音被称为自然导音. 和声小调与和声大调一样, 主要特点是第VI级与第VII级间的增二度音程. 图式第二小节为a和声小调.\par
\qquad 3. 旋律小调: 由升高自然小调中的第VI级和第VII级而成, 一般多用于上行(下行中也偶尔可见). 旋律小调在下行时多用自然小调的形式(即第VI级和第VII级还原). 图示第三小节为a旋律小调(上行), 第四小节示a旋律小调下行.\par
\startextract
\Notes \wh{abcdefgh} \en\bar
\Notes \wh{hijklm} \sh n\wh {no} \en\bar
\Notes \wh{hijkl} \sh m\wh m\sh n\wh {no}\en
\zendextract
\startextract
\Notes \wh o \na n\wh n \na m\wh m \wh{lkjih} \en
\zendextract

\clearpage

\begin{center}
 \textbf{2.大小调式音级关系}\\
\end{center}

[\textbf{大小调式音级}] 按音阶上行计算, 大小调式音级的名称如下.\par
\qquad 第I级: 主音(T);\par
\qquad 第II级: 上主音(下行导音);\par
\qquad 第III级: 中音, 在主音和属音之间;\par
\qquad 第IV级: 下属音(S), 在主音下方纯五度;\par
\qquad 第V级: 属音(D), 在主音上方纯五度;\par
\qquad 第VI级: 下中音, 在主音和下属音之间;\par
\qquad 第VII级: 导音, 向上倾向于主音.\par
\qquad 下图将主音绘制在最中间, 依次为: IV, VI, VII, I, II, III, V音级, 其中第一个音级下属音(S)与主音(T)之间以及主音(T)与最后一个音级属音(V)之间均为纯五度.\par
\startextract
\Notes \wh{fhijkln}\en
\zendextract

[\textbf{正音级与副音级}] 调式中第I级, 第IV级和第V级称为正音级, 其他音级为副音级.\par

[\textbf{大小调式体系中的稳定度}] 大小调式中, 起稳定作用的是I, III, V级. 其中I级最稳定, III级和V级稳定性较差.\par
\qquad 三个稳定音级和它们的稳定性只有和主音三和弦共响时才能表现出来, 假使用其他非主音三和弦, 则不具有稳定性.\par
\qquad 第II, IV, VI和VII级是不稳定音级, 在适当条件下, 它们显露出以二度关系进行到稳定音的倾向.\par
\qquad 下图中奇数位置的音级均以二度关系进行到相邻的偶数位置音级. 顺序为$VII\rightarrow I\leftarrow II\rightarrow III \leftarrow IV \rightarrow V\leftarrow VI$.\par
\startextract
\Notes \wh{bcdefgh}\en
\zendextract
\qquad 备注: 在大小调体系中, 不稳定音到稳定音的倾向基本上是完全一致的, 只是倾向的尖锐程度有所差别.\par

[\textbf{不稳定到稳定倾向的尖锐程度}] 造成不稳定音到稳定音倾向的尖锐程度的差别主要有两种原因.\par
\qquad 1. 由于稳定音的不同而不同. 如: 大调中 II-I 比II-III要尖锐, 因为I比III更稳定.\par
\qquad 2. 由于不稳定音到稳定音的距离而不同, 即半音倾向比全音倾向尖锐. 如大调中的 VI-V 比 II-I尖锐, 大调中的 VI-V 和 II-III没有小调中的 VI-V 和 II-III 尖锐等.\par

\clearpage

[\textbf{大小调式体系的判别图}] \par
\qquad $\begin{cases} \mbox{\textbf{大调式}: 主音与中音的大三度}\begin{cases}\mbox{\textbf{自然大调}}\begin{cases}\mbox{\textbf{自然大调}}\\ \mbox{\textbf{利底亚调}: 升高下属音} \\ \mbox{\textbf{混合利底亚调}: 降低导音}\end{cases}\\ \mbox{\textbf{和声大调}: 降低下中音构成, 下中音与导音的增二度}\\ \mbox{\textbf{旋律大调}: 下行级进并降低下中音和导音}\end{cases}\\ \\ \mbox{\textbf{小调式}: 主音与中音的小三度}\begin{cases}\mbox{\textbf{自然小调}} \begin{cases}\mbox{\textbf{自然小调}}\\ \mbox{\textbf{多利亚调}: 升高下中音}\\ \mbox{\textbf{弗里几亚调}: 降低上主音}\end{cases}\\ \mbox{\textbf{和声小调}: 升高导音, 下中音与导音的增二度}\\ \mbox{\textbf{旋律小调}: 升高下中音和导音}\end{cases}\end{cases}$

\end{document}
