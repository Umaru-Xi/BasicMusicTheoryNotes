\documentclass[a4paper,20pt]{article}
\usepackage{geometry}
\usepackage{savesym}
\usepackage{amsmath}
\usepackage{esint}
\usepackage{mathrsfs}
\usepackage{amssymb}

\usepackage[UTF8]{ctex}
\savesymbol{breve}
\usepackage{musixtex}
\restoresymbol{breve}{breve}

\input{musixlit}


\geometry{left=1.5cm, right=1.5cm, top=1.5cm, bottom=1.5cm}
\setlength{\lineskip}{0.75em}
\setlength{\parskip}{0.75em}

\begin{document}
\begin{center} 
 \Large \textbf{音乐理论基础}\par
 \textbf{十一.和弦}
\end{center}

\large 
\begin{center}
 \textbf{1.五声调式体系中的和弦}\\
\end{center}

[\textbf{七声调式中的三和弦}] 每一种七声调式都有一个大三和弦, 一个小三和弦和一个减三和弦.\par
\qquad 以C宫体系为例;\par
\qquad 1. 清乐音阶: \par
\qquad \qquad 大三和弦: 宫, 清角, 徵;\qquad 小三和弦: 商, 角, 羽;\qquad 减三和弦: 变宫;\par
\startextract
\Notes \zsong{宫} \zh{eg}\wh c \en
\Notes \zsong{商} \zh{fh}\wh d \en
\Notes \zsong{角} \zh{gi}\wh e \en
\Notes \zsong{清角} \zh{hj}\wh f \en
\Notes \zsong{徵} \zh{ik}\wh g \en
\Notes \zsong{羽} \zh{jl}\wh h \en
\Notes \zsong{变宫} \zh{km}\wh i\en
\zendextract
\qquad 2. 雅乐音阶:\par
\qquad \qquad 大三和弦: 宫, 商, 徵; \qquad 小三和弦: 角, 羽, 变宫;\qquad 减三和弦: 清角;\par
\startextract
\Notes \zsong{宫} \zh{eg}\wh c \en
\Notes \zsong{商} \sh f\zh{fh}\wh d \en
\Notes \zsong{角} \zh{gi}\wh e \en
\Notes \zsong{变徵} \sh f\zh{hj}\wh f \en
\Notes \zsong{徵} \zh{ik}\wh g \en
\Notes \zsong{羽} \zh{jl}\wh h \en
\Notes \zsong{变宫} \sh m\zh{km}\wh i \en
\zendextract
\qquad 3. 燕乐音阶:\par
\qquad \qquad 大三和弦: 宫, 清角, 闰;\qquad 小三和弦: 商, 徵, 羽;\qquad 减三和弦: 角;\par
\startextract
\Notes \zsong{宫} \zh{eg}\wh c \en
\Notes \zsong{商} \zh{fh}\wh d \en
\Notes \zsong{角} \fl i\zh{gi}\wh e \en
\Notes \zsong{清角} \zh{hj}\wh f \en
\Notes \zsong{徵} \fl i\zh{ik}\wh g \en
\Notes \zsong{羽} \zh{jl}\wh h \en
\Notes \zsong{闰} \fl i\zh{km}\wh i \en
\zendextract

[\textbf{调式中和弦的名称和标记}]\par
\qquad 1. 在五声调式体系中, 和弦的名称以根音为基础, 如宫音上构成的和弦为宫和弦;\par
\qquad 2. 调式中的和弦也用调式音级的号数来表示, 如第I级上的和弦用I来表示;\par
\qquad 3. 对于第I级, 第V级和第IV级, 还可以分别用主(T), 属(D)和下属(S)来表示;\par
\qquad 另外, 为了清楚表示和弦类别, 大三和弦和增三和弦用大写字母标记, 小三和弦和减三和弦用小写字母标记.\par

\clearpage

\begin{center}
 \textbf{2.大小调体系中的和弦}\\
\end{center}

[\textbf{大小调体系中的三和弦}]\par
\qquad 1. 自然大调: 谱示C自然大调;\par
\qquad \qquad 大三和弦: I, IV, V级;\qquad 小三和弦: II, III, VI级;\qquad 减三和弦: VII级;\par
\startextract
\Notes \zsong{I} \zh{eg}\wh c \en
\Notes \zsong{II} \zh{fh}\wh d \en
\Notes \zsong{III} \zh{gi}\wh e \en
\Notes \zsong{IV} \zh{hj}\wh f \en
\Notes \zsong{V} \zh{ik}\wh g \en
\Notes \zsong{VI} \zh{jl}\wh h \en
\Notes \zsong{VII} \zh{km}\wh i \en
\zendextract
\qquad 2. 自然小调: 谱示a自然小调;\par
\qquad \qquad 大三和弦: III, VI, VII级;\qquad 小三和弦: I, IV, V级;\qquad 减三和弦: II级;\par
\startextract
\Notes \zsong{I} \zh{jl}\wh h \en
\Notes \zsong{II} \zh{km}\wh i \en
\Notes \zsong{III} \zh{eg}\wh c \en
\Notes \zsong{IV} \zh{fh}\wh d \en
\Notes \zsong{V} \zh{gi}\wh e \en
\Notes \zsong{VI} \zh{hj}\wh f \en
\Notes \zsong{VII} \zh{ik}\wh g \en
\zendextract
\qquad 3. 和声大调: 谱示C和声大调;\par
\qquad \qquad 大三和弦: I, V级;\qquad 小三和弦: III, IV级;\qquad 减三和弦: II, VII级;\par
\startextract
\Notes \zsong{I} \zh{eg}\wh c \en
\Notes \zsong{II} \fl h\zh{fh}\wh d \en
\Notes \zsong{III} \zh{gi}\wh e \en
\Notes \zsong{IV} \fl h\zh{hj}\wh f \en
\Notes \zsong{V} \zh{ik}\wh g \en
\Notes \zsong{VI} \fl h\zh{jl}\wh h \en
\Notes \zsong{VII} \zh{km}\wh i \en
\zendextract
\qquad 4. 和声小调: 谱示a和声小调;\par
\qquad \qquad 大三和弦: I, IV级;\qquad 小三和弦: V, VI级;\qquad 减三和弦: II, VII级;\par
\startextract
\Notes \zsong{I} \zh{jl}\wh h \en
\Notes \zsong{II} \zh{km}\wh i \en
\Notes \zsong{III} \sh n\zh{ln}\wh j \en
\Notes \zsong{IV} \zh{fh}\wh d \en
\Notes \zsong{V} \sh g\zh{gi}\wh e \en
\Notes \zsong{VI} \zh{hj}\wh f \en
\Notes \zsong{VII} \sh g\zh{ik}\wh g \en
\zendextract

[\textbf{正三和弦和副三和弦}] \par
\qquad 正三和弦: 第I级(主音), 第V级(属音)和第IV级(下属音)上的和弦因为具有重大意义, 所以被称为正三和弦;\par
\qquad 副三和弦: 第II, III, VI, VII级上的和弦.\par

[\textbf{属七和弦}] 第V级(属音)上的七和弦.\par
\qquad 属七和弦是不协和和弦, 属七和弦的解决: 进行到主音三和弦;\par
\qquad \qquad 1. 第VII级(导音)和第II级(上主音)进行到第I级(主音);\par
\qquad \qquad 2. 第IV级(下属音)进行到第III级(中音);\par
\qquad \qquad 3. 第V级(属音)在原位属七和弦中进行到第I级(主音), 在转位属七和弦中保持原位不动.\par

\clearpage

[\textbf{导七和弦}] 第VII级(导音)上的七和弦.\par
\qquad 导七和弦是不协和和弦, 导七和弦的解决: 进行到重复三度音的主音三和弦;\par
\qquad \qquad 1. 第VII级(导音)进行到第I级(主音);\par
\qquad \qquad 2. 第II级(上主音)和第IV级(下属音)进行到第III级(中音);\par
\qquad \qquad 3. 第VI级(下中音)进行到第V级(属音).\par

\begin{center}
 \textbf{3.和弦的应用及其表现特性}\\
\end{center}

[\textbf{多声部音乐中的和弦应用}] 较多使用的是协和三和弦(即大小三和弦), 其次是增减三和弦和七和弦等不协和和弦.\par

[\textbf{大三和弦的表现特性}] 具有明亮的色彩, 由根音上的大三度造成.\par
\qquad 大三和弦和大三度音程一样, 多说明大调的特征.\par

[\textbf{小三和弦的表现特性}] 色彩比较暗淡, 由根音上的小三度造成.\par
\qquad 小三和弦和小三度音程一样, 多说明小调的特征.\par

[\textbf{增三和弦}] 具有向外扩张的特征, 且具有一切不协和和弦的尖锐和紧张的特点.\par

[\textbf{减三和弦}] 具有向内收缩的特征, 且具有一切不协和和弦的尖锐和紧张的特点.\par

[\textbf{完全和弦}] 和弦中的音全部出现时叫做完全和弦.\par
\qquad 在和弦实际应用中, 往往在不同的八度中重复其全部音或其中的某几个音.\par

[\textbf{不完全和弦}] 和弦中的音没有全部出现叫做不完全和弦.\par
\qquad 在二声部乐曲中, 和弦音一定要被省略. 为了辨认这些和弦, 需要在想象中添补上所省略的音.\par

[\textbf{和弦的作用}] 和弦可以用来作为旋律的伴奏, 或将许多声部组织在一起, 并也可以出现在旋律本身之中(即旋律按照和弦音进行).\par

\end{document}
