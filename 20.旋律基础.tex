\documentclass[a4paper,20pt]{article}
\usepackage{geometry}
\usepackage{savesym}
\usepackage{amsmath}
\usepackage{esint}
\usepackage{mathrsfs}
\usepackage{amssymb}

\usepackage[UTF8]{ctex}
\savesymbol{breve}
\usepackage{musixtex}
\restoresymbol{breve}{breve}

\input{musixlit}


\geometry{left=1.5cm, right=1.5cm, top=1.5cm, bottom=1.5cm}
\setlength{\lineskip}{0.75em}
\setlength{\parskip}{0.75em}

\begin{document}
\begin{center} 
 \Large \textbf{音乐理论基础}\par
 \textbf{二十.旋律基础}
\end{center}

\large 
\begin{center}
 \textbf{1.旋律}\\
\end{center}

[\textbf{旋律}] 用调式关系和节奏节拍关系组合起来的, 具有独立性的许多音的单声部进行, 被称为旋律. 旋律分为声乐旋律和器乐旋律两种.\par
\qquad 声乐旋律: 为人声演唱的. 声乐旋律的音域比较狭窄, 富于歌唱性是其最大特点;\par
\qquad 器乐旋律: 为乐器演奏的. 与乐器的特点有着紧密联系, 并以乐器的不同而有所不同. 一般比声乐旋律音域宽, 速度和力度变化较大, 富于节奏型和技巧性.\par

[\textbf{主调音乐}] 在多声部音乐中, 旋律以不同的方式组合. 以一个声部为主, 其他声部为副(往往具有伴奏性质), 被称为主调音乐.\par

[\textbf{复调音乐}] 两个以上具有独立意义的旋律, 协调地结合在一起, 被称为复调音乐.\par

\begin{center}
 \textbf{2.旋律发展的方法}\\
\end{center}

[\textbf{旋律发展的主要方法}] 重复, 变化, 模进, 扩展, 紧缩.\par

[\textbf{重复}] 重复可以使旋律在发展中更加巩固并得到统一. \par
\qquad 重复可以分为: 原样重复和变化重复.\par

[\textbf{变化}] 变化可以使旋律增加新的表现因素, 一般多是在统一的节奏基础上将音高加以改变, 但也可能在音高上节奏上都加以改变.\par

[\textbf{模进}] 模进也被称为不同高度的重复, 模进有同调模进和转调模进两种.\par
\qquad 同调模进: 改变调式音级但不改变调性;\par
\qquad 转调模进: 改变调性不改变调式音级.\par

[\textbf{扩展}] 在音程上或节奏上扩展.\par

[\textbf{紧缩}] 在音程上或节奏上紧缩.\par

\begin{center}
 \textbf{3.旋律进行与高潮}\\
\end{center}

[\textbf{旋律进行的方向}] 旋律进行的方向和紧张度以及力度有密切的关系. 旋律进行的方向要在总提上和局部两个方面分析.\par
\qquad 1. 横向进行: 音的重复;\par
\qquad 2. 上行: 旋律紧张度增长或力度增强;\par
\qquad 3. 下行: 旋律紧张度缓和或力度减弱;\par
\qquad 4. 波浪式进行.\par

[\textbf{旋律的顶点}] 旋律高涨时所达到的最高的音被称为顶点.

\clearpage

[\textbf{高潮}] 当旋律的最高处与乐曲情绪的最紧张处相复合时, 称其为高潮.\par

\begin{center}
 \textbf{4.旋律分段}\\
\end{center}

[\textbf{旋律的段落}] 旋律并不是连贯不断地进行的, 而是分成若干个相互联系的部分, 这些部分被称为段落.\par

[\textbf{旋律的停顿}] 段落与段落之间的瞬间分割, 被称为停顿.\par
\qquad 停顿的标志:\par
\qquad \qquad 1. 休止符;\par
\qquad \qquad 2. 较长音的停留之后;\par
\qquad \qquad 3. 节奏型的重复之间.\par

[\textbf{段落的终止}] 用来结束段落的几个音或某几个和弦的序列, 被称为终止. 终止总是位于停顿之前.\par
\qquad 终止的基本类型:\par
\qquad \qquad 1. 完全终止: 旋律结束在主三和弦的根音上;\par
\qquad \qquad 2. 不完全终止: 旋律结束在主三和弦的三度音或五度音上;\par
\qquad \qquad 3. 半终止: 旋律结束在不稳定的音级上, 或结束在用属三和弦或属七和弦伴奏的第V级上.\par

[\textbf{乐段}] 能够表达完整或相对完整的乐思, 用完全终止来结束的具有相当独立性的段落, 被称为乐段.\par
\qquad 乐段一般由两部分组成, 两部分往往开始相同, 而结束不同. 有的乐段甚至只有最后一个音不同.\par

[\textbf{乐句}] 乐段中的主要组成部分. 乐段中一般都包含两个乐句. \par
\qquad 1. 第一乐句: 多半用半终止或不完全终止来结束, 造成乐思的不完整性和尚待继续发展的状况;\par
\qquad 2. 第二乐句: 差不多总是用完全终止来结束.\par

[\textbf{不转调乐段}] 乐曲结束时仍保持原调.\par

[\textbf{转调乐段}] 乐曲结束时发生了转调.\par

[\textbf{乐节}] 有的乐句可以分成两个较小的部分被称为乐节; \par

[\textbf{动机}] 乐节的进一步划分被称为动机, 是旋律组织的最小单位且至少包含一个强音和一个弱音.\par
\qquad 动机的长度一般为一个小节. 两个动机构成一个乐节, 两个乐节形成一个乐句, 两个乐句形成一个乐段. 因此一个乐段往往是八个小节.\par
\qquad 注意: 并非所有乐段都有这样划分的可能.\par

\clearpage

\begin{center}
 \textbf{5.乐曲的基本形式}\\
\end{center}

[\textbf{乐曲的最简单形式}] 一段体, 二段体和三段体.\par
\qquad 一段体: 以一个乐段构成的乐曲形式, 一般只有一个音乐形象, 一个意境. 但结构简单精炼, 最易被广大群众所接受;\par
\qquad 二段体: 由两个对比的性质不同的乐段构成的乐曲形式. 许多带有副歌的群众歌曲是由二段体写成的.\par
\qquad \qquad 二段体的第一段通常具有陈述性质, 结束是不稳定的,要求乐曲继续发展;\par
\qquad \qquad 二段体的第二段通常承接第一段的乐意, 作更进一步的发挥, 在情感上更加浓烈和奔放. \par
\qquad 三段体: 由三个乐段构成的乐曲形式. 三段体的作品整体上更加统一和完整, 在器乐小品中最为多见.\par
\qquad \qquad 三段体的第三段通常是第一段的重复, 重复时可以完全不变或稍加改变;\par
\qquad \qquad 三段体的第二段被称为中段, 与第一和第三段成为明显的对比. \par

[\textbf{复合曲式}] 多个二段体和三段体组合在一起, 形成较为复杂的复合曲式. 复合曲式多在较大型的音乐作品中出现.\par

\end{document}
