\documentclass[a4paper,20pt]{article}
\usepackage{geometry}
\usepackage{savesym}
\usepackage{amsmath}
\usepackage{esint}
\usepackage{mathrsfs}
\usepackage{amssymb}

\usepackage[UTF8]{ctex}
\savesymbol{breve}
\usepackage{musixtex}
\restoresymbol{breve}{breve}

\input{musixlit}


\geometry{left=1.5cm, right=1.5cm, top=1.5cm, bottom=1.5cm}
\setlength{\lineskip}{0.75em}
\setlength{\parskip}{0.75em}

\begin{document}
\begin{center} 
 \Large \textbf{音乐理论基础}\par
 \textbf{一.音及音高}
\end{center}

\large 
\begin{center}
 \textbf{1.音}\\
\end{center}

[\textbf{乐音的产生}] 音乐中所使用的音, 是人们在长期的生产斗争和阶级斗争中为了表现自己的生活和思想感情而特意挑选出来的.\par

[\textbf{音的性质}] 音有高低, 强弱, 长短, 音色等四个性质.\par
\qquad 1. 音的高低(主要): 由物体振动的频率确定;\par
\qquad 2. 音的长短(主要): 由物体振动的时间确定;\par
\qquad 3. 音的强弱(次要): 由物体振动的幅度确定;\par
\qquad 4. 音的音色(次要): 由物体的性质和形状等属性确定.\par

[\textbf{乐音与噪音}] 规则的音为乐音, 不规则的音为噪音. 噪音在音乐表现中也具有不可缺少的作用.\par

[\textbf{基音}] 发声物体全段振动所对音的音.\par

[\textbf{泛音}] 发声物体各部分共同振动所产生的音, 是一种复合音.\par

\begin{center}
 \textbf{2.音高}\\
\end{center}

[\textbf{乐音体系}] 音乐中使用的有固定音高的音的总和.\par

[\textbf{音列}] 将乐音体系中的音按照上行或下行次序排列起来称为音列.\par

[\textbf{音级}] 乐音体系中的各个音被称为音级, 音级有基本音级和变化音级两种.\par
\qquad 1. 基本音级: 具有独立名称的音级称为基本音级(与钢琴上的白键对应);\par
\qquad 2. 变化音级: 由基本音级升高或降低而得来的音级(与钢琴上的黑键对应).\par

[\textbf{基本音级记号}] 基本音级有字母和唱名两种记号方式.\par
\qquad 1. 字母记号: C\quad d\qquad e\quad f\qquad g\quad a\quad b(h)\quad;\par
\qquad 2. 唱名记号: do\quad re\quad mi\quad fa\quad sol\quad la\quad si.\par

[\textbf{变化音级}] \par
\qquad 1. 用记号``\#"标记将基本音级升高半音, 如$^\# C$;\par
\qquad 2. 用记号''b"标记将基本音级降低半音, 如$^b C$;\par
\qquad 3. 用记号``$\times$"标记将基本音级升高全音, 如$^\times C$;\par
\qquad 4. 用记号''bb"标记将基本音级降低全音, 如$^{bb} C$.\par

[\textbf{八度}] 两个相邻的具有同样名称的音称为八度.\par

\clearpage

[\textbf{音的分组}] 由低到高:\par
\qquad 大字二组: $C_2\quad D_2\quad E_2\quad F_2\quad G_2\quad A_2\quad B_2$;\par
\qquad 大字一组: $C_1\quad D_1\quad E_1\quad F_1\quad G_1\quad A_1\quad B_1$;\par
\qquad 大字组: \quad $C\quad D\quad E\quad F\quad G\quad A\quad B$;\par
\qquad 小字组: \quad $c\quad d\quad e\quad f\quad g\quad a\quad b$;\par
\qquad 小字一组: $c^1\quad d^1\quad e^1\quad f^1\quad g^1\quad a^1\quad b$;\par
\qquad 小字二组: $c^2\quad d^2\quad e^2\quad f^2\quad g^2\quad a^2\quad b$;\par
\qquad 小字三组: $c^3\quad d^3\quad e^3\quad f^3\quad g^3\quad a^3\quad b$;\par
\qquad 小字四组: $c^4\quad d^4\quad e^4\quad f^4\quad g^4\quad a^4\quad b$;\par
\qquad 小字五组: $c^5$.\par

[\textbf{音域}] 音列的范围, 有总的音域和个别的音域两种.\par
\qquad 1. 总的音域: 音列的总范围, 从$C_2$到$c^5$;\par
\qquad 2. 个别的音域: 发声物体所能达到的部分, 如钢琴为$A_2$到$c^5$

[\textbf{音区}] 将音域分为低音区, 中音区和高音区三部分, 其中每一部分为一个音区.\par
\qquad 1. 低音区: 大字组, 大字一组和大字二组;\par
\qquad 2. 中音区: 小字组, 小字一组和小字二组;\par
\qquad 3. 高音区: 小字三组, 小字四组和小字五组.\par

\end{document}
