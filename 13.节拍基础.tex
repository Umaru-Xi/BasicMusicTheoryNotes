\documentclass[a4paper,20pt]{article}
\usepackage{geometry}
\usepackage{savesym}
\usepackage{amsmath}
\usepackage{esint}
\usepackage{mathrsfs}
\usepackage{amssymb}

\usepackage[UTF8]{ctex}
\savesymbol{breve}
\usepackage{musixtex}
\restoresymbol{breve}{breve}

\input{musixlit}


\geometry{left=1.5cm, right=1.5cm, top=1.5cm, bottom=1.5cm}
\setlength{\lineskip}{0.75em}
\setlength{\parskip}{0.75em}

\begin{document}
\begin{center} 
 \Large \textbf{音乐理论基础}\par
 \textbf{十三.节拍基础}
\end{center}

\large 
\begin{center}
 \textbf{1.单节拍}\\
\end{center}

[\textbf{单节拍}] 每小节有两个或三个单位拍的节拍.\par
\qquad 两个单位拍的节拍应用在 $\frac{2}{2}, \frac{2}{4}, \frac{2}{8}$中;\par
\qquad 三个单位拍的节拍应用在 $\frac{3}{2}, \frac{3}{4}, \frac{3}{8}$中.\par

[\textbf{单拍子}] 每小节有两个或三个单位拍的拍子.\par

[\textbf{组合法}] 把小节中的音符按照节拍的结构划分成音群的方法.\par
\qquad 使用组合法的目的在于便于视谱和易于辨认各种节奏型.\par

[\textbf{单拍子的音值组合法}]\par
\qquad 1. 基本单位拍之间应该用不连结的符尾彼此分开. 因此在单位拍中, 有几个单位拍就有几个音群. 在单位拍的音值少于四分音符且节奏不复杂时, 往往用共同符尾连结小节中的所有单位. 为了清晰, 第二条和第三条的符尾可以根据单位拍分开;\par
\qquad 2. 在节奏划分比较繁琐的情况下, 每个主要的音群可以再分成两个或四个相等的附属音群. 第一条符尾可不分开;\par
\qquad 3. 包括全小节的音值, 用一个音符标记, 不遵守单位拍彼此分开的原则;\par
\qquad 4. 休止符也可以按照组合法的规则来组合, 但不需要连线. 整小节的休止, 用全休止符标记;\par
\qquad 5. 附点音符可以不遵守每个单位拍必须要分开的原则;\par
\generalmeter{\meterfrac34}
\startextract
\Notes \ha e\ha e \en\bar
\Notes \qa e\Dqbu ee\ha e\en\bar
\Notes \ibu0e0\qbp0e\roff{\tbbu0\tqh0e} \ibu0e0\qb0e\nbbbu0\qb0e\nbbbu0\qb0e\nbbbu0\qb0e\tbu0\qb0e \qa e\en\bar
\Notes \ha {.e}\en\bar
\Notes \qa {.e} \ca e \qa e\en
\zendextract
\qquad 6. 当单位拍中不止一个音符, 但最后一个音符带有附点并占用下一拍时间时, 附点因其明显型不足而不用.\par
\generalmeter{\meterfrac24}
\startextract
\Notes \Dqbu {e}{.e} \Tqbbu eee \en\bar
\Notes = \ibu0e0\qb0e\tbu0\qb0e \islurd0e \en \Notes \tslur0e \Qqbbu eeee \en
\zendextract

\clearpage

\begin{center}
 \textbf{2.复节拍}\\
\end{center}

[\textbf{复节拍}] 用同类的单节拍合成序列, 因而不止一个重音的节拍.\par
\qquad 复节拍中的重音数目与合成复节拍的单节拍数目相同.\par
\qquad 由两拍的单节拍组成的复节拍应用在 $\frac{4}{2}, \frac{4}{4}, \frac{4}{8}$ 中;\par
\qquad 由三拍的单节拍组成的复节拍应用在 $\frac{6}{4}, \frac{6}{8}, \frac{6}{16}, \frac{9}{4}, \frac{9}{8}, \frac{9}{16}, \frac{12}{8}, \frac{12}{16}$ 中.\par

[\textbf{复拍子}] 用同类的单拍子合成序列, 因而不止一个重音的拍子.\par
\qquad 复拍子中的重音数目与合成复拍子的单拍子的数目相同.\par

[\textbf{强拍}] 在复节拍(或复拍子)中第一个重音较后面的重音强, 称为强拍. 其他带重音的单位拍称为次强拍.\par

[\textbf{复拍子的音值组合法}] \par
\qquad 1. 合成复拍子的单拍子应该彼此明显分开, 每个单拍子根据单拍子的组合法;\par
\qquad 2. 包括全小节的音值, 用一个音符记写, 与单拍子一样. 但九个单位的复拍子不能用一个音符记写.\par
\generalmeter{\meterfrac44}
\startextract
\Notes \ha e\ha e\en\bar
\Notes \qa e\qa e\qa e\qa e\en\bar
\Notes \ha e \Qqbu eeee\en
\zendextract
\startextract \barno=4
\Notes \ibu0e0\qb0e\nbbu0\qb0e\tbu0\qb0e \Qqbbu eeee \Dqbu ee \qa e\en\bar
\Notes \qa {.e} \ca e \ha e\en\bar
\Notes \wh e\en
\zendextract

[\textbf{混合复节拍}] 由二拍的单节拍和三拍的单节拍按照不同的次序结合成序列.\par
\qquad 五个单位的混合复节拍和混合复拍子应用在 $\frac{5}{4}, \frac{5}{8}$ 中;
\qquad 七个单位的混合复节拍和混合复拍子应用在 $\frac{7}{4}$ 中;\par
\qquad 其他情况的混合复节拍和混合复拍子还见于 $\frac{11}{4}$ 中.\par

[\textbf{混合复拍子}] 由二拍的单拍子和三拍的单拍子按照不同的次序结合成序列.\par

[\textbf{混合复拍子组合次序的表记}]\par
\qquad 1. 用虚线分隔音符;\par
\qquad 2. 用连线连结音符;\par
\qquad 3. 用括号中的拍号, 如将拍号写为: $\frac{5}{4} \left(\frac{3}{4}+\frac{2}{4}\right)$.\par

[\textbf{混合复拍子的音值组合法}] 使所有的单拍子彼此分开, 其中单拍子根据单拍子音值组合法进行组合.\par

\end{document}
