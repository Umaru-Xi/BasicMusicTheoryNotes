\documentclass[a4paper,20pt]{article}
\usepackage{geometry}
\usepackage{savesym}
\usepackage{amsmath}
\usepackage{esint}
\usepackage{mathrsfs}
\usepackage{amssymb}

\usepackage[UTF8]{ctex}
\savesymbol{breve}
\usepackage{musixtex}
\restoresymbol{breve}{breve}

\input{musixlit}


\geometry{left=1.5cm, right=1.5cm, top=1.5cm, bottom=1.5cm}
\setlength{\lineskip}{0.75em}
\setlength{\parskip}{0.75em}

\begin{document}
\begin{center} 
 \Large \textbf{音乐理论基础}\par
 \textbf{七.和弦基础}
\end{center}

\large 
\begin{center}
 \textbf{1.三和弦与七和弦}\\
\end{center}

[\textbf{和弦}] 在多声部音乐中, 可以按照三度关系排列起来的三个以上的音的结合称为和弦.\par
\qquad 按照三度关系构成的和弦, 由于各音间保持着一定的紧张密度, 音响协调丰满, 并且合乎泛音的自然规律;\par
\qquad 备注:部分作品中可能存在不是按三度关系构成的和弦.\par
\startextract
\Notes \zw{ce}\wh g\en\bar
\Notes \zw{df}\wh h\en\bar
\Notes \zw{eg}\wh i\en
\zendextract

[\textbf{三和弦}] 三个音按照三度关系堆叠起来的和弦.\par
\qquad 根音(一度音): 三和弦中最下面的音, 用数字1来代表;\par
\qquad 三度音: 中间的音, 用数字3来代表;\par
\qquad 五度音: 最上面的音, 用数字5来代表.\par

[\textbf{大三和弦}] 根音到三度音是大三度, 三度音到五度音是小三度, 根音到五度音是纯五度.\par
\startextract
\Notes \zw{ce}\wh g\en\bar
\Notes \sh g\zw{eg}\wh i\en\bar
\Notes \fl h\fl l\zw{hj}\wh l\en
\zendextract

[\textbf{小三和弦}] 根音到三度音是小三度, 三度音到五度音是大三度, 根音到五度音是纯五度.\par
\startextract
\Notes \zw{ce}\wh g\en\bar
\Notes \sh g\zw{eg}\wh i\en\bar
\Notes \fl h\fl l\zw{hj}\wh l\en
\zendextract

[\textbf{增三和弦}] (不常用)根音到三度音和三度音到五度音均是大三度, 根音到五度音是增五度.\par
\startextract
\Notes \sh g\zw{ce}\wh g\en\bar
\Notes \fl h\zw{hj}\wh l\en\bar
\Notes \sh m\fl i\zw{mi}\wh k\en
\zendextract

[\textbf{减三和弦}] (不常用)根音到三度音和三度音到五度音均是小三度, 根音到五度音是减五度.\par
\startextract
\Notes \zw{ik}\wh m\en\bar
\Notes \sh j\zw{jl}\wh n\en\bar
\Notes \fl l\zw{hj}\wh l\en
\zendextract

\clearpage

[\textbf{协和音程与不协和音程}] 听起来悦耳的音程称为协和音程; 听起来刺耳或彼此不融合的音程称为不协和音程.\par
\qquad 大小三和弦都是协和和弦, 因为其中包含的音程都是协和音程(大三度, 小三度, 纯五度);\par
\qquad 增减三和弦都是不协和和弦, 因为其中的减五度和增五度是不协和音程.\par

[\textbf{七和弦}] 四个音按照三度关系堆叠起来的和弦称为七和弦.\par
\qquad 七和弦下面的三个音与三和弦一样, 叫做: 根音, 三度音, 五度音;\par
\qquad 七度音: 第四个音, 用数字7代表. 七和弦由此得名.\par
\qquad 注意: 所有七和弦都是不协和和弦, 因为包含了不协和的七度音程.\par

[\textbf{七和弦的构成}]\par
\qquad 大小七和弦(大调小七和弦): 以大三和弦为基础, 根音到七度音为小七度的七和弦;\par
\qquad 小小七和弦(小七和弦): 以小三和弦为基础, 根音到七度音是小七度的七和弦;\par
\qquad 减小七和弦(半减七和弦): 以减三和弦为基础, 根音到七度音是小七度的七和弦;\par
\qquad 减减七和弦(减七和弦): 以减三和弦为基础, 根音到七度音是减七度的七和弦;\par
\qquad 此外还有: 增大七和弦, 大大七和弦, 小大七和弦等.\par
\startextract
\Notes \zw{gik}\wh m\en\bar
\Notes \zw{dfh}\wh j\en\bar
\Notes \zw{ikm}\wh o\en\bar
\Notes \sh g\zw{gik}\wh m\en\bar
\Notes \fl h\zw{hjl}\wh n\en\bar
\Notes \zw{fhj}\wh l\en\bar
\Notes \fl h\zw{fhj}\wh l\en
\zendextract

\begin{center}
 \textbf{2.原位和弦与转位和弦}\\
\end{center}

[\textbf{原位和弦}] 以根音为低音的和弦称为原位和弦.\par

[\textbf{转位和弦}] 以三度音, 五度音, 七度音为低音的和弦称为转位和弦.\par
\qquad 三和弦除了根音外还有两个音, 因此三和弦有两个转位;\par
\qquad \qquad (1). 六和弦: 以三度音为低音的三和弦称为三和弦的第一转位;\par
\qquad \qquad (2). 四六和弦: 以五度音为低音的三和弦称为三和弦的第二转位;\par
\qquad 七和弦除了根音之外还有三个音, 所以七和弦有三个转位;\par
\qquad \qquad (1). 五六和弦: 以三度音为低音的七和弦称为七和弦的第一转位;\par
\qquad \qquad (2). 三四和弦: 以五度音为低音的七和弦称为七和弦的第二转位;\par
\qquad \qquad (3). 二和弦: 以七度音为低音的七和弦称为七和弦的第三转位.\par
\startextract
\Notes \zw{eg}\wh j\en\bar
\Notes \zw{gj}\wh l\en\bar
\Notes \zw{ikn}\lw m\en\bar
\Notes \zw{dfi}\rw g\en\bar
\Notes \zw{fik}\rw g\en
\zendextract

\clearpage

[\textbf{构成和识别和弦}] \par
\qquad 方法一: 记住和弦的音程结构, 按音程结构去构成和识别和弦;\par
\qquad 方法二: 首先确定和弦的根音, 根据根音构成或找出和弦的原位形式, 再根据原位和弦来构成或识别转位和弦.\par

[\textbf{等和弦}] 两个和弦孤立起来听具有同样的声音效果, 但在音乐中的意义不同, 写法也不同, 这样的和弦称为等和弦. 等和弦有两种.\par
\qquad 1. 和弦中的音不因为等音变化而改变音程的结构;\par
\startextract
\Notes \lsh e\sh c\sh g\zw{ce}\wh g \lfl d\fl h\zw{df}\wh h\en\bar
\Notes \lsh e\sh c\sh g\zw{ceg}\wh i \lfl j\fl d\fl h\zw{dfh}\wh j\en
\zendextract
\qquad 2. 由于等音变化而改变和弦的结构.\par
\startextract
\Notes \sh g\zw{gik}\wh m \lfl h\lw h\zw{ik}\wh m\en\bar
\Notes \sh j\zw{fh}\wh j \fl k\zw{fh}\wh k\en
\zendextract


\end{document}
