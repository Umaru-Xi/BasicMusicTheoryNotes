\documentclass[a4paper,20pt]{article}
\usepackage{geometry}
\usepackage{savesym}
\usepackage{amsmath}
\usepackage{esint}
\usepackage{mathrsfs}
\usepackage{amssymb}

\usepackage[UTF8]{ctex}
\savesymbol{breve}
\usepackage{musixtex}
\restoresymbol{breve}{breve}

\input{musixlit}


\geometry{left=1.5cm, right=1.5cm, top=1.5cm, bottom=1.5cm}
\setlength{\lineskip}{0.75em}
\setlength{\parskip}{0.75em}

\begin{document}
\begin{center} 
 \Large \textbf{音乐理论基础}\par
 \textbf{十七.调式变音及半音阶}
\end{center}

\large 
\begin{center}
 \textbf{1.调式变音}\\
\end{center}

[\textbf{调式变音}] 将七声自然调式中的基本音级加以半音变化(升高或降低)所获得的音, 被称为调式变音.\par
\qquad 调式变音由基本音级变化而来, 故仍然按照调式基本音级记谱, 但带上临时变音记号.\par
\qquad 调式变音可以在任何音级上产生, 但一般以小二度关系倾向稳定音级较为典型.\par
\qquad 自然调式中, 某些音级的固定半音变化会产生独立的调式. 如: 和声大小调和旋律大小调.\par
\qquad 五声调式不具有半音进行的旋律特点, 所以带有变音的五声调式比较少见.\par

[\textbf{辅助音}] 两个同音高的音中间所插入的上方二度或下方二度的音. 辅助音有在自然音阶之内的自然辅助音和不在自然音阶内的变化辅助音.\par
\qquad 五声音阶中, 辅助音可能与主要音构成小三度, 如: 徵角徵, 宫羽宫等.\par

[\textbf{经过音}] 在两个音高不同的音中间插入的音, 在进行中一般不超过全音. 经过音包括在自然音阶之内的自然经过音和不在自然音阶内的变化经过音.\par
\qquad 在五声音阶中, 可能包括小三度进行. 因为小三度在五声音阶中算级进音程.\par

[\textbf{调式变音的进行}] 调式变音一般以半音形式进行. 过程中有: 导音的形成(变化半音), 导音的解决(自然半音), 导音的消失(变化半音)等情况.\par
\qquad 导音的消失: 形成反向的新的导音;\par
\qquad 下谱一小节示: 上行过程中导音形成-导音解决-导音形成-导音解决;
\qquad 下谱二小节示: 下行过程中音符进行-导音消失-导音解决-音符进行-音符进行-导音形成-导音解决.\par
\startextract
\Notes \wh c \sh c\wh c \wh d \sh d\wh d \wh e\en\bar
\Notes \wh j \wh i \fl i\wh i \wh h \wh g \sh f\wh f \na f\wh f \wh e\en
\zendextract

[\textbf{调式变音和转调的区别}] 调式变音中的变音记号不引起调的转移, 而转调中的变音记号则标志着新调的建立.\par

\clearpage

\begin{center}
 \textbf{2.半音阶}\\
\end{center}

[\textbf{半音阶}] 由半音组成的音阶.\par
\qquad 半音阶不是一种独立的音阶, 而是在七音级自然调式的大二度之间填入半音而成的.\par

[\textbf{半音阶的书写方法}] 以调性关系关系为依据, 当调式变音和半音音阶具有旋律的性质并且能清晰听出相邻两个音的相互关系的结合时, 记谱上则应表明导音的形成, 解决, 消失. \par
\qquad 1. 在大调内: 音阶中所有的调式基本音级不加更动;\par
\qquad \qquad 上行时大二度间用升I级, 升II级, 升IV级, 升V级, 降VII级来填补; \par
\qquad \qquad 下行时大二度间用降VII级, 降VI级, 降III级, 降II级, 升IV级来填补.\par
\qquad 2. 在小调内: \par
\qquad \qquad 上行时按照平行大调(关系大调)记谱;\par
\qquad \qquad 下行时按照同名大调记谱.\par
\qquad 3. 以五声音阶为基础的半音阶非常罕见.\par

[\textbf{半音阶的另一种书写方法}] 当调式变音和半音阶包括在滑奏的半音进行和技术性的经过中时, 两个相邻音的相互关系的结合不可能被听出来, 此时只有进行的方向具有主要意义. 采用此书写方法.\par
\qquad 上行时大二度之间用升高下面的音来填补;\par
\qquad 下行时大二度之间用降低上面的音来填补.\par

[\textbf{变音在音乐表现中的作用}]\par
\qquad 变音的表现力是自然音体系中所不具有的;\par
\qquad 在带有歌词的歌曲中, 变音使音乐音调和语言音调紧密结合, 自然而协调地表达了歌中的情绪.\par

\end{document}
