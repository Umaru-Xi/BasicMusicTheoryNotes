\documentclass[a4paper,20pt]{article}
\usepackage{geometry}
\usepackage{savesym}
\usepackage{amsmath}
\usepackage{esint}
\usepackage{mathrsfs}
\usepackage{amssymb}

\usepackage[UTF8]{ctex}
\savesymbol{breve}
\usepackage{musixtex}
\restoresymbol{breve}{breve}

\input{musixlit}


\geometry{left=1.5cm, right=1.5cm, top=1.5cm, bottom=1.5cm}
\setlength{\lineskip}{0.75em}
\setlength{\parskip}{0.75em}

\begin{document}
\begin{center} 
 \Large \textbf{音乐理论基础}\par
 \textbf{十四.节奏与节拍}
\end{center}

\large 
\begin{center}
 \textbf{1.变换节拍\quad 交错节拍\quad 自由节拍\quad $\frac{1}{4}$拍子}\\
\end{center}

[\textbf{变换节拍}] 在乐曲中各种节拍交替出现.\par

[\textbf{变换拍子}] 在乐曲中各种拍子交替出现.\par
\qquad 变换拍子的变换可能是有规律地循环出现, 也可能是不规律的循环.\par

[\textbf{变换拍子的记写}] \par
\qquad 正规变换拍子: 用并列拍号一次标明;\par
\qquad 不正规的变换拍子: 在乐曲开始变换前加以标明.\par
\generalmeter{\meterfrac44 \meterfrac34}
\startextract
\Notes \Dqbu gh \Qqbbu ddde \Dqbu gj \Qqbbu hhhg\en
\zendextract
\generalmeter{\meterfrac48}
\startextract
\Notes \ibu0d0\qbp0d\roff{\tbbu0\tqh0e} \qa f\en
\generalmeter{\meterfrac38}\changecontext
\notesp \qa {.f}\en
\generalmeter{\meterfrac48}\changecontext
\notesp \ca g \ca h \ca g \ca f\en
\zendextract

[\textbf{交错节拍}] 各种不同节拍的同时结合. 交错节拍有两种情况:\par
\qquad 1. 每小节中的强拍是相符合的;\par
\qquad 2. 部分强拍相符合, 但部分强排不相符合.\par

[\textbf{交错拍子}] 各种不同拍子的同时结合.\par
\qquad 交错拍子一般很少使用, 如在三个乐队演奏三种不同节拍乐曲的记谱中.\par

[\textbf{自由节拍}] 节拍的重音和单位拍的时值都不是非常明显固定的, 由表演者根据乐曲的内容和个人的体会自由处理, 也被称为散板.\par
\qquad 自由节拍一般不记拍号, 有时会用``サ''来标记.

[\textbf{$\frac{1}{4}$拍子}] $\frac{1}{4}$拍子是少见的拍子, 完全用$\frac{1}{4}$拍写成的乐曲非常少, 其常在变换拍子中出现. \par
\qquad 在戏曲音乐的急板中常出现.\par

[\textbf{板眼}] 在我国民间音乐中, 常用``板''和``眼''来标记节拍和拍子. 其中``板''表示强拍, ``眼''表示弱拍和次强拍.\par
\qquad 二拍子: 一板一眼;\qquad 三拍子: 一板二眼;\qquad 四拍子: 一板三眼;\par
\qquad 八拍子: 加赠板一板三眼;\qquad 一拍子: 有板无眼.\par
\qquad 眼又分为头眼, 二眼, 中眼, 末眼等.\par

\clearpage

[\textbf{常用板眼符号}]按照第一拍到第八拍的顺序记写.\par
\qquad 拍\qquad 数: \quad (头)板\quad (头)眼\quad 中(二)眼\quad 末眼\quad 腰板\quad 头眼\quad 中眼\quad 末眼\par
\qquad 有板无眼: \quad 、($\times$)\par
\qquad 一板一眼: \quad 、($\times$)\qquad $\cdot$\par
\qquad 一板二眼: \quad 、($\times$)\qquad $\cdot$\qquad $\cdot$\par
\qquad 一板三眼: \quad 、($\times$)\qquad $\cdot$\qquad $\circ$\qquad $\cdot$\par
\qquad 加赠板的\par
\qquad 一板三眼: \quad 、($\times$)\qquad $\cdot$\qquad $\circ$\qquad $\cdot$\qquad $\times$($\times$)\qquad $\cdot$\qquad $\circ$\qquad $\cdot$\par

\begin{center}
 \textbf{2.声乐曲中的音值组合法}\\
\end{center}

[\textbf{声乐曲中影响音值组合的因素}] 歌词.\par

[\textbf{声乐曲中的音值组合法}]\par
\qquad 1. 一字一音时: 符尾应分开写, 不和其他音符组合在一起;\par
\qquad 2. 一字数音时: 按照音值组合法的规则进行组合, 并用连线把配一个字的数个音符连接起来.\par
\qquad 特殊情况: 为了表示适当的音乐分句及乐曲中某一种固定的节奏型, 有时可不遵守组合法中单位拍彼此分开的原则, 甚至可以把两个小节中的音符用共同的符尾连结起来.\par

\end{document}
