\documentclass[a4paper,20pt]{article}
\usepackage{geometry}
\usepackage{savesym}
\usepackage{amsmath}
\usepackage{esint}
\usepackage{mathrsfs}
\usepackage{amssymb}

\usepackage[UTF8]{ctex}
\savesymbol{breve}
\usepackage{musixtex}
\restoresymbol{breve}{breve}

\input{musixlit}


\geometry{left=1.5cm, right=1.5cm, top=1.5cm, bottom=1.5cm}
\setlength{\lineskip}{0.75em}
\setlength{\parskip}{0.75em}

\begin{document}
\begin{center} 
 \Large \textbf{音乐理论基础}\par
 \textbf{五.音程基础与五声调式}
\end{center}

\large 
\begin{center}
 \textbf{1.音程基础}\\
\end{center}

[\textbf{音程}] 两个音级在音高上的相互关系.\par

[\textbf{旋律音程}] 先后弹奏的两个音形成旋律音程.\par

[\textbf{和声音程}] 同时弹奏的两个音形成和声音程.\par

[\textbf{旋律音程和和声音程的书写}] \par
\qquad 1. 旋律音程书写时要错开;\par 
\qquad 2. 和声音程书写时要上下对齐. 和声二度中, 低音在左高音在右, 两个音紧靠在一起.\par
\startextract
\Notes \wh g\wh h\en
\Notes \wh e\wh g\en\bar
\Notes \lw{g}\wh{h}\en
\Notes \zw{e}\wh{g}\en
\zendextract

[\textbf{根音与冠音}] 音程中下面的音为根音, 上面的音为冠音.\par

[\textbf{旋律音程的分类}] 旋律音程依照进行方向分为: 上行, 下行和平行.\par

[\textbf{音程的读法}]\par
\qquad 1. 和声音程的读法: 由根音读至冠音;\par
\qquad 2. 上行旋律音程的读法: 由根音读至冠音;\par
\qquad 3. 下行旋律音程和平行旋律音程读时说明方向, 如: C到平行的C, D到下面的G.\par

[\textbf{音程的组成}] 音程由级数(度数)和音数组成.\par
\qquad 1. 级数: 音程在五线谱上所包含的线与间的数目, 也叫度数. 如: 同一线上或间内构成的音程为一度, 相邻的线与间构成的音程为二度, 相邻的线或间构成的音程为三度等;\par
\qquad 2. 音数: 音程中包含的半音或全音的数目;\par
\qquad 3. 区分级数相同而音数不同的音程, 用: 大, 小, 增, 减, 倍增, 倍减, 纯等.\par

[\textbf{自然音程}] 纯音程, 大音程, 小音程, 增四度和减五度(合成三整音或三全音). 它们产生于其他音程之前, 也称基本音程.\par
\qquad 一, 四, 五, 八无大小音程. 二, 三, 六, 七无纯音程.\par
\qquad (提示: 基本音级除E-F, B-C外均为全音, E-F和B-C为半音)\par
\qquad 1. 纯一度: 音数为$0$的一度. 如: C到C, $^\#D$到$^\#D$;\par
\qquad 2. 小二度: 音数为$\frac{1}{2}$的二度. 如: E到F, $^\#F$到G;\par
\qquad 3. 大二度: 音数为$1$的二度. 如: C到D;\par
\clearpage
\qquad 4. 小三度: 音数为$1\frac{1}{2}$的三度. 如: E到G;\par
\qquad 5. 大三度: 音数为$2$的三度. 如: C到E;\par
\qquad 6. 纯四度: 音数为$2\frac{1}{2}$的四度. 如: C到F;\par
\qquad 7. 增四度: 音数为$3$的四度. 如: F到B;\par
\qquad 8. 减五度: 音数为$3$的五度. 如: B到F;\par
\qquad 9. 纯五度: 音数为$3\frac{1}{2}$的五度. 如: C到G;\par
\qquad 10. 小六度: 音数为$4$的六度. 如: E到C;\par
\qquad 11. 大六度: 音数为$4\frac{1}{2}$的六度. 如: C到A;\par
\qquad 12. 小七度: 音数为$5$的七度. 如: G到F;\par
\qquad 13. 大七度: 音数为$5\frac{1}{2}$的七度. 如: C到B;\par
\qquad 14. 纯八度: 音数为$6$的八度. 如: $C^1$到$C^2$.\par

[\textbf{变化音程}] 一切增减音程和倍增与倍减音程称为变化音程. (增四度和减五度除外).\par
\qquad 将冠音升高或将根音降低可以增加音数, 将冠音降低或将根音升高可以减少音数.\par
\qquad 1. 增音程: 大音程和纯音程增大半音;\par
\qquad 2. 减音程: 小音程或纯音程减少半音(没有减一度);\par
\qquad 3. 大音程: 小音程增大半音;\par
\qquad 4. 小音程: 大音程减少半音;\par
\qquad 6. 倍增音程: 增音程增大半音. 常用倍增一度, 倍增四度和倍增八度;\par
\qquad 7. 倍减音程: 减音程减小半音. 常用倍减五度和倍减八度.\par

[\textbf{单音程}] 不超过八度的音程. 称为: 单音程某某度\par

[\textbf{复音程}] 超过八度的音程. 相当于单音程加上一个到四个纯八度而成. 除某些独立名称外(下面列出), 称为: 隔开几个八度的某某度.\par
\qquad 1. 九度: 隔一个八度的二度;\par
\qquad 2. 十度: 隔一个八度的三度;\par
\qquad 3-7. 十一到十四度: 隔一个八度的四到七度;\par
\qquad 8. 十五度: 隔开一个八度的八度.\par

\clearpage

\begin{center}
 \textbf{2.五声调式}\\
\end{center}

[\textbf{五声调式的定义}] 依照纯五度排列起来的五个音所构成的调式.\par
\qquad 五个音依次是: 宫, 徵, 商, 羽, 角.\par
\startextract
\Notes \wh c \wh g \wh k \wh o \wh s \en
\zendextract

[\textbf{五声调式的特点}] 即钢琴上黑键的排列方式.\par
\qquad 1. 缺少半音和三整音这类音程的尖锐倾向, 宫角之间形成五声调式中唯一的大三度(或小六度);\par
\qquad 2. 以大二度和小三度所构成的``三音组"是五声调式旋律进行中的基础音调;\par

[\textbf{三音组}] 五声调式可以被认为是由两个相同或不同的三音组结合而成的, 三音组有三种: 大二度大二度, 大二度小三度, 小三度大二度.\par
\qquad 下谱依照上面的顺序排列. \par
\startextract
\Notes \wh c\wh d\wh e\en\bar
\Notes \wh g\wh h\wh j\en\bar
\Notes \wh h\wh j\wh k\en\bar
\zendextract

[\textbf{五声调式}] 以''宫徵商羽角"中的某个音为主音就称之为某调式, 如宫调式.\par
\qquad 下谱中各小节按照``宫调式, 徵调式, 商调式, 羽调式, 角调式"的顺序排列.\par
\startextract
\Notes \wh c\wh d\wh e \en
\Notes \wh g\wh h\wh j\en\bar
\Notes \wh g\wh h\wh j\en
\Notes \wh k\wh l\wh n\en\bar
\Notes \wh d\wh e\wh g\en
\Notes \wh h\wh j\wh k\en
\zendextract
\startextract
\barno=4
\Notes \wh h\wh j\wh k\en
\Notes \wh l\wh n\wh o\en\bar
\Notes \wh e\wh g\wh h\en
\Notes \wh j\wh k\wh l\en
\zendextract
\qquad 宫调式: 大二度大二度+大二度小三度, 小三度连接, ''-3/";\par
\qquad 徵调式: 大二度小三度+大二度小三度, 大二度连接, ``/2/'';\par
\qquad 商调式: 大二度小三度+小三度大二度, 大二度连接, ``/2$\backslash$";\par
\qquad 羽调式: 小三度大二度+小三度大二度, 大二度连接, ''$\backslash$2$\backslash$";\par
\qquad 角调式: 小三度大二度+大二度大二度, 小三度连接, ``$\backslash$3-''.\par

\clearpage

[\textbf{五声调式的特征}] 相邻两调式的差异发生在``宫"和''角"两音上.\par
\startextract \Notes \wh d\wh e\sh{f}\wh f \wh h\wh i\wh k\en\zendextract
\startextract \Notes \wh d\wh e\wh g\wh h\wh i\wh k\en\zendextract
\startextract \Notes \wh d\wh e\wh g\wh h\wh j\wh k\en\zendextract
\startextract \Notes \wh d\wh f\wh g\wh h\wh j\wh k\en\zendextract
\startextract \Notes \wh d\wh f\wh g\fl{i}\wh i\wh j\wh k\en\zendextract
\qquad 宫调式: 主音上的大三度;\par
\qquad 徵调式: 主音上的纯四度和大六度;\par
\qquad 商调式: 主音上的大二度和小七度;\par
\qquad 羽调式: 主音上的小三度和纯五度;\par
\qquad 角调式: 主音上的主音上的小六度.\par
\qquad 相差1-4个调号的两个五声调式有1-4个音不同.\par

[\textbf{五声调式音级的特性}]\par
\qquad 1. 在单声部音乐中, 起稳定作用的是第I级(主音). 其次是与主音成四或五度关系的第V级和第IV级, 虽然稳定性较差, 但对主音有较大的支持力;\par
\qquad 2. 多声部音乐中, 五声调式音级的稳定性由调式和声配置决定;\par
\qquad 3. 五声调式中除了稳定音级外, 其他各音都是不稳定的, 分别以大二度或小三度倾向于稳定音, 构成五声调式进行的基本音调.\par



\end{document}
