\documentclass[a4paper,20pt]{article}
\usepackage{geometry}
\usepackage{savesym}
\usepackage{amsmath}
\usepackage{esint}
\usepackage{mathrsfs}
\usepackage{amssymb}

\usepackage[UTF8]{ctex}
\savesymbol{breve}
\usepackage{musixtex}
\restoresymbol{breve}{breve}

\input{musixlit}


\geometry{left=1.5cm, right=1.5cm, top=1.5cm, bottom=1.5cm}
\setlength{\lineskip}{0.75em}
\setlength{\parskip}{0.75em}

\begin{document}
\begin{center} 
 \Large \textbf{音乐理论基础}\par
 \textbf{二.音律}
\end{center}

\large 
\begin{center}
 \textbf{1.音律}\\
\end{center}

[\textbf{音律}] 乐音体系中各音的绝对准确高度及相互关系.\par

[\textbf{十二平均律}] 将八度分成十二个均等的部分(半音)的音律.\par
\qquad \quad C\quad $^\#C$\quad D\quad $^\#D$\quad E\quad F\quad $^\#F$\quad G\quad $^\#G$\quad A\quad $^\#A$\quad B\par
\qquad 1. 全音: 距离等于两个半音.\par
\qquad 2. 在基本音级中, 除了E-F和B-C是半音距离外, 其他相邻两音都是全音;\par
\qquad 3. 钢琴上相邻的两个琴键构成半音, 隔开一个琴键的两个音构成全音;\par
\qquad 4. 特点: 在前后结合和同时结合上不够纯正自然, 但转调方便, 在键盘乐器的演奏和制造上具有许多优势, 被广泛使用.\par

[\textbf{等音}] 音高相同但记法不同的音, 如: $^\#C$ = $^bD$ = $^\times B$.\par
\qquad 除$^\#G$ = $^bA$只有一个等音以外, 每个音级都有两个等音.\par

[\textbf{其他律制}] 五度相生律和纯律.\par
\qquad 1. 五度相生律: 前后结合自然协调, 适合单音音乐;\par
\qquad 2. 纯律: 和弦音结合时纯正和谐, 适合多声音乐.\par
\qquad 3. 它们的缺点: 不易频繁转调, 难以在键盘乐器上演奏.\par

\begin{center}
 \textbf{2.自然音和变化音}\\
\end{center}

[\textbf{自然半音}] 两个相邻的音级构成的半音, 如: e-f, $^\#e$-$^\#f$, $^\#g$-a.\par

[\textbf{自然全音}] 两个相邻的音级构成的全音, 如: c-d, $^\#c$-$^\#d$, e-$^\#f$.\par

[\textbf{变化半音}] 同一音级的两种不同形式构成的半音, 如: c-$^\#c$, d-$^bd$, $^\#c-^\times c$.\par

[\textbf{变化全音}] 同一音级的两种不同形式构成的全音.\par


\end{document}
