\documentclass[a4paper,20pt]{article}
\usepackage{geometry}
\usepackage{savesym}
\usepackage{amsmath}
\usepackage{esint}
\usepackage{mathrsfs}
\usepackage{amssymb}

\usepackage[UTF8]{ctex}
\savesymbol{breve}
\usepackage{musixtex}
\restoresymbol{breve}{breve}

\input{musixlit}


\geometry{left=1.5cm, right=1.5cm, top=1.5cm, bottom=1.5cm}
\setlength{\lineskip}{0.75em}
\setlength{\parskip}{0.75em}

\begin{document}
\begin{center} 
 \Large \textbf{音乐理论基础}\par
 \textbf{十八.移调}
\end{center}

\large 
\begin{center}
 \textbf{1.移调}\\
\end{center}

[\textbf{移调}] 为了使一支旋律适合不同的人声和乐器以及创作的需要, 将旋律由一个调移至另一个调. 这种将作品的全部或作品的一部分的由一个调移至另一个调的情况, 称为移调.\par

[\textbf{移调的应用种类}]\par
\qquad 1. 在创作乐曲时, 为了使旋律得到进一步的拓展, 在不同高度上重复着旋律;\par
\qquad 2. 由于人声音域的不同, 要使一些为高音所创作的歌曲适合于低音演唱, 需要移调;\par
\qquad 3. 为某些移调乐器作曲时, 应进行移调记谱. 乐谱上的移调永远和器乐上的移调方向相反.\par

[\textbf{移调的方法}]\par
\qquad 1. 按照音程关系移调:\par
\qquad \qquad 首先, 明确作品的原调和要移到的新调, 或明确要移高还是移低的音程度数;\par
\qquad \qquad 然后, 在谱号旁写上新的调号, 把作品中的每一个音按照已确定的音程度数移动, 在移动中要保持音数的准确(如保持变音记号);\par
\qquad \qquad 最后, 不要仅限于机械的纯粹的音程移动, 要用听觉在理论上掌握音及和弦的调式意义.\par
\qquad 2. 更改调号的移调(移至增一度):\par
\qquad \qquad 把所有的音符保持在原位不动, 只用新的调号来代替原调的调号, 新调只能与原调相距半音;\par
\qquad \qquad 如果旋律中有临时记号, 则应记上新的相应记号, 保持原记号的升高或降低的含义.\par
\qquad 3. 更改谱号的移调: 为了演奏过程中的移调而使用, 只适合于读谱号不困难的人.\par
\qquad \qquad 首先, 在五线谱上找到原调的主音, 将该音改称为新调的主音;\par
\qquad \qquad 其次, 从各种谱号中找出适合这个音符新意义的谱号;\par
\qquad \qquad 最后, 在新谱号旁记上新的调号. 如果有临时变音记号, 则记以新的变音记号以保持原升高和降低的意义不变.\par

\end{document}
