\documentclass[a4paper,20pt]{article}
\usepackage{geometry}
\usepackage{savesym}
\usepackage{amsmath}
\usepackage{esint}
\usepackage{mathrsfs}
\usepackage{amssymb}

\usepackage[UTF8]{ctex}
\savesymbol{breve}
\usepackage{musixtex}
\restoresymbol{breve}{breve}

\input{musixlit}


\geometry{left=1.5cm, right=1.5cm, top=1.5cm, bottom=1.5cm}
\setlength{\lineskip}{0.75em}
\setlength{\parskip}{0.75em}

\begin{document}
\begin{center} 
 \Large \textbf{音乐理论基础}\par
 \textbf{三.记谱法}
\end{center}

\large 
\begin{center}
 \textbf{1.音符与休止符}\\
\end{center}

[\textbf{记谱法}] 记录乐曲的方法.\par

[\textbf{五线谱}] 用来记载音符的五条平行横线.\par
\startextract \zendextract
\qquad 1. 间: 自下而上的线间空隙, 分别为1,2,3,4间(超过部分为``上加"或''下加"间);\par
\qquad 2. 线: 自下而上的线, 分别为1,2,3,4,5线(超过部分为``上加"或''下加"线).\par

[\textbf{音符}] 记录不同长短的音的进行的符号.\par
\startextract \Notes \wq{'c} \wh{c} \hu{c} \hl{c} \qu{c} \ql{c} \cu{c} \cl{c} \ccu{c} \ccl{c} \cccu{c} \cccl{c} \ccccu{c} \ccccl{c} \en\zendextract
\qquad 1. 二全音符; \qquad 2. 全音符;\qquad 3. 二分音符(2个);\qquad 4. 四分音符(2个);\par
\qquad 5. 八分音符(2个);\qquad 6. 十六分音符(2个);\qquad 7. 三十二分音符(2个);\par
\qquad 8. 六十四分音符(2个).\par

[\textbf{休止符}] 记录不同长短的音的间断的符号.\par
\startextract \Notes \PAuse \pause \hp \qp \ds \qs \hs \qqs \en\zendextract
\qquad 1. 二全休止符;\qquad 2. 全休止符;\qquad 3. 二分休止符;\qquad 4. 四分休止符;\par
\qquad 5. 八分休止符;\qquad 6. 十六分休止符;\qquad 7. 三十二分休止符;\par
\qquad 8. 六十四分休止符.\par

[\textbf{连谱号}] 用于连接多行五线谱, 由起线和括线两部分组成.\par
\instrumentnumber{1}
\setstaffs1{2}
\startextract\zendextract
\qquad 1. 起线: 连结多个五线谱的垂直线;\par
\qquad 2. 括线: 连结多个五线谱的括弧(或直线).\par

\clearpage

[\textbf{音值}] 音的长度. 较大的音值与次之的音值之比为2:1.\par

[\textbf{括线}] 括线是连结多个五线谱的括弧或直线.\par
\instrumentnumber3
\setstaffs12
\akkoladen{{2}{3}}
\startextract\zendextract
\qquad 1. 花括线: 连结钢琴, 风琴, 手风琴, 竖琴, 扬琴, 琵琶等乐器;\par
\qquad 2. 直括线: 合奏,合唱或乐队使用. 用来连结同类乐器, 并将它们分组. 有时在直括线上还加入辅助线(花或直)来连接同种乐器;\par
\qquad 3. 在总谱中, 独唱或独奏声部如果之包括一行或两行谱线, 则左边只画一条起线而不加括线.\par

\begin{center}
 \textbf{2.音符与休止符的写法}\\
\end{center}

[\textbf{音符的组成}] 音符由符头(空心或实心的椭圆), 符干(垂直的短线)和符尾(连在符干一段的旗帜形标记)组成.\par

[\textbf{音符的符头}] 符头记在五线谱的线上或间内, 符头在五线谱的位置越高音高越高.\par

[\textbf{音符的符干}] \par
\qquad 1. 当符头在第三线以上时, 符干朝下, 写在符头左边;\par
\qquad 2. 当符头在第三线以下时, 符干朝上, 写在符头右边;\par
\qquad 3. 当符头在第三线上时, 符干方向由临近的符干方向确定;\par
\qquad 4. 当一个符干连接多个符头时, 以距离第三线最远的符头为准.\par

[\textbf{音符的符尾}] 符尾永远记在符干右边并弯向符头方向.\par
\qquad 当多个音符组成一组共用一个符尾时, 称为符杠.\par

[\textbf{符杠}] 多个音符组成一组时, 可以共用符尾相连. 此时符干方向以距离第三线最远的符头为准. 多个符干要平行.\par
\instrumentnumber{1}
\akkoladen{}
\setstaffs1{1}
\startextract \Notes
\qu{e} \qu{g} \cu{i} \ql{m} \ql{k} \cl{i} \zq{j}\ql{l} \zq{km}\ql{o} \zq{df}\qu{h} \zq{gi}\qu{k} \zq{gi}\ql{k} \zq{ej}\qu{l} \Dqbl hj \Tqbl hkm \Qqbbl ghgl
\en\zendextract 

\clearpage

[\textbf{多声部与单声部符干记法}]\par
\qquad 1. 单声部音乐永远用单符干记谱;\par
\qquad 2. 多声部音乐在节奏相同情况下才用单符干记谱. 节奏不同时, 高声部符干向上, 低声部符干向下, 符干交错时原则依然不变.\par

[\textbf{符干的长度}] 符干的长度一般应保持八度音程距离. 若多个音符通过符杠连接, 则符杠距离符头最近距离应至少八度.\par

[\textbf{休止符记法}] \par
\qquad 1. 单独干记谱中, 休止符永远记在第三线上或靠近第三线的地方;\par
\qquad 2. 双符干记谱中, 各声部共同休止时与单符干相同;\par
\qquad 3. 双符干记谱中, 个别声部休止时休止符记在五线谱边缘或五线谱外(二分休止符在加线上面, 全休止符在加线下面).\par

[\textbf{附点记法}] 附点写在符头或休止符右侧的间内, 不记在线上.\par
\startextract \Notes
\qu{e} \qu{.e} \qu{i} \qu{.i} \ql{k} \ql{.k} \ds \dsp \hp \hpp
\en\zendextract 

\begin{center}
 \textbf{3.谱号}\\
\end{center}

[\textbf{谱号}] 用于标记五线谱音高标准的符号.\par

[\textbf{G谱号}] 表示小字一组的g, 也称谓高音谱号. 一般记在第二线上. 记在第一线时为古法国式高音谱号.\par
\setclef1\treble
\startextract 
\Notes \wh{g}
\en\zendextract 

[\textbf{F谱号}] 表示小字组f, 也称低音谱号, 记在第四线上. 记在第五线上为倍低音谱号.\par
\setclef1\bass
\startextract 
\Notes \wh{M}
\en\zendextract 

[\textbf{C谱号}] 表示小字一组的c, 可以记在五线谱上任意一条线上.\par
\setclef1\alto
\startextract 
\Notes \wh{c}
\en\zendextract 

\clearpage

\begin{center}
 \textbf{4.增长音值的补充记号}\\
\end{center}

\setclef1\treble
\startextract 
\Notes \ccu{g} \wh{.i} \qlp{k} \clpp{m} \hp \dsp \qppp \en\bar
\Notes \isluru0j\wh{j} \tslur0j\ql{j} \en\bar
\Notes \fermataup{k}\qlp{i} \fermataup{k}\ds \fermataup{k}\en\bar
\Notes \qlp{i} \qp \fermataup{k}\en\doublebar
\Notes \ccu{f} \ds \fermataup{k}\en\rightrepeat
\zendextract 

[\textbf{附点}] 带有一个附点的音符增长原有音符的二分之一, 带有两个附点则增长四分之三.\par

[\textbf{延音线}] 用在音高相同的两个或两个以上音符上时, 表示它们为一个音, 音的长度为它们的总和.\par
\qquad 1. 单声部音乐中, 连线永远写在与符干相反的方向;\par
\qquad 2. 两声部音乐中, 高声部连线朝上弯曲, 低声部连线朝下弯曲;\par
\qquad 3. 多声部音乐中, 连线分写在上下两侧.\par

[\textbf{延长号}] 表示根据实际情况可以自由地增长音符或休止符的时值.\par
\qquad 1. 在单声部音乐中, 延长号写在音符或休止符的上面;\par
\qquad 2. 多声部音乐中, 延长号可以写在音符或休止符的下面;\par
\qquad 3. 延长号记在小节线上, 表示小节之间休息片刻;\par
\qquad 4. 延长号记在双纵线上, 表示乐曲的结束或告一段落.\par

\begin{center}
 \textbf{5.变音记号}\\
\end{center}

\setclef1\treble
\startextract 
\Notes \sh{j}\en\bar
\Notes \fl{f}\en\bar
\Notes \dsh{m}\en\bar
\Notes \dfl{j}\en\bar
\Notes \na{i}\en
\zendextract 

[\textbf{变音记号}] 表示升高或降低基本音级的记号, 按顺序记在图例中前五小节.\par
\qquad 1. 升记号: 表示将基本音级升高半音;\par
\qquad 2. 降记号: 表示将基本音级降低半音;\par
\qquad 3. 重升记号: 表示将基本音级升高两个半音(一个全音);\par
\qquad 4. 重降记号: 表示将基本音级降低两个半音(一个全音);\par
\qquad 5. 还原记号: 表示将已经升高或降低的音还原.\par

[\textbf{调号}] 记在谱号后面的变音记号, 在改变调号之前其对音列中所有同音名的音都生效.\par
\qquad 1. 更换调号发生在一行乐谱开始处, 则在前一行结尾的小节线后将要更换的调号写出;\par
\qquad 2. 要增加原有变音记号的数目, 只需要在小节线的右边写出新的调号即可;\par
\qquad 3. 减少原有变音记号的数目, 则需要在小节线左边将要删除的变音记号还原;\par
\qquad 4. 变更原有变音记号, 应该现在小节线左边将其还原, 再在小节线右边写出新的调号.\par

\clearpage

[\textbf{临时记号}] 写在音符前, 且只作用于同一小节同音高的音符. 在多声部音乐中只对一个声部有效. 有时为了提醒废除原有的变音记号, 会在小节线后加上另外的临时记号.\par

\begin{center}
 \textbf{6.省略记号}\\
\end{center}

[\textbf{移动八度记号}] 记在五线谱上面时表示在虚线范围内升高八度, 在谱线下面表示降低八度.\par
\startextract 
\Notes \octfinup{10}{3}\qu{i} \qu{g} \qu{e}\en\bar
\Notes \octfindown{-5}{3}\ql{i} \ql{k} \ql{m}\en
\zendextract 

[\textbf{重复八度记号}] 将数字``8''记在音符上面或下面, 表示该音要同时发出高八度或低八度的声音. 当较长时间需要重复八度时, 可以在移动八度记号前加``Con".\par

[\textbf{长休止记号}] 记在第三线上, 表示休止对音的小节数.\par
\startextract 
\Notes \Hpause 41 4 \en
\Notes \en\bar
\zendextract 

[\textbf{震音记号}] 表示音或和弦均匀地快速交替.\par
\qquad 下面的记号中, 记号与奏法交替出现.\par
\startextract 
\Notes \trml{j}\hl{j} \Qqbl jjjj  \en\bar
\Notes \trrmu{f}\hu{f} \Qqbbu ffff \en\bar
\Notes \ibbl0k2 \hl{i} \tbu0 \hl{k} \Qqbbl ikik\en
\zendextract 
\qquad 1. 一个音或和弦的震音记号, 震音总时值与音符的时值相同:\par
\qquad \qquad (1). 全音符在三线以上时, 震音记号在三线以下, 反之;\par
\qquad \qquad (2). 比全音符时值小的音符, 斜线贯穿符干. 若有符尾则与符尾平行(斜线算符尾).\par
\qquad 2. 两个音或和弦的震音记号斜线在两个音或和弦之间靠近记写符尾的地方, 震音总时值与一个音或和弦相同.\par

[\textbf{反复记号}] 乐曲部分或全部重复.\par
\startextract 
\Notes \duevolte \en\bar
\Notes \en\setvoltabox{1}\leftrepeat
\Notes \en\setendvoltabox\setvolta{2}\rightrepeat
\Notes \en\setendvolta\bar
\Notes \segno e\en\bar
\Notes \coda e\en
\zendextract 
\qquad 1. 某一音型重复时用斜线标记, 斜线数目与符尾数目相同(大概是''/``的形状);\par
\qquad 2. 一次或多次重复某一小节时, 用上图第一小节的记号表示. 若该记号记在两小节之间的小节线上, 则表示前面两小节旋律型再重复一次;\par
\qquad 3. 乐曲中较大的重复用第三小节的反复记号. 如果从头重复, 则可省略左端记号;\par
\qquad 4. 如果重复时结尾不同, 则在第三小节基础上增加上面的符号和第四小节的记号来标明;\par
\qquad (接下页)
\clearpage
\qquad 5. 如果乐曲由三部分组成, 第三部分是第一部分的重复, 则在第二部分结尾写上''D.C.
", 并在第一部分结尾处写``Fine". 如果不是第一部分从头开始重复, 则在开始处记第五小节的记号;\par
\qquad 6. 如果重复部分后面有一个相当大的结尾, 则在进入结尾的地方和结尾开始地方记录第六小节的记号, 并在结尾开始处注明<从''第五小节的记号"奏至``第六小节的记号"然后结尾>;\par 

\begin{center}
 \textbf{7.演奏法的记号}\\
\end{center}

[\textbf{连音奏法}] 在连线内, 不同音高的音要素要连贯, 连音奏法大部分记在五线谱上面, 很少记在下面.\par
\startextract 
\Notes \isluru0l\ql{j} \ql{l} \Dqbu eg \Dqbl ik \en\bar
\Notes \Qqbu jkmn \Qqbl fehg\en\bar
\Notes \tslur0l\ql{i}\en
\zendextract

[\textbf{断音奏法}] 表示某些音或和弦要断续地弹奏, 有三种记法如下图.\par
\qquad 1. 在单声部音乐中, 断音记号通常记在符头那边;\par
\qquad 2. 一行五线谱上记有两个声部并不用相同的符干时, 分别记在相反的方向.\par
\qquad 下面的记号中, 记号与奏法交替出现.\par
\startextract
\Notes \upz{j}\ql{j} \lpz e\qu e \uppz{g}\ql{g} \isluru0l\upz{i}\ql{i}\tslur0l\upz{k}\ql{k}\en\bar
\Notes \cl j\ds \cu e\ds \ccl{g}\dsp \clp{i}\qs \clp{k}\qs\en
\zendextract

[\textbf{持续音奏法}] \par
\qquad 1. 第一小节记法中, 持续音记号表示该音稍强奏并充分保持该音的值;\par
\qquad 2. 第二小节记法中, 持续音记号表示该音稍强奏, 但各音稍分离.\par
\startextract
\Notes \ust{j}\ql{j} \lst{f}\qu{f}\en\bar
\Notes \upzst{k}\ql{k} \lpzst{e}\qu{e}\en
\zendextract

[\textbf{滑音奏法}] 一种有特色的民间音乐奏法.\par
\qquad 1. 第一小节表示向上滑音;\par
\qquad 2. 第二小节表示向下滑音.\par
\startextract
\Notes \ql{i}\slide i33\en
\Notes \en\bar
\Notes \ql{i}\slide {i}{3}{-3}\en
\Notes \en
\zendextract

[\textbf{琶音奏法}] 将各和弦中的音由下而上很快地分散弹奏.\par
\startextract
\Notes \arpeggio{g}{4}\zq{hj}\ql{m}\en
\zendextract


\end{document}
