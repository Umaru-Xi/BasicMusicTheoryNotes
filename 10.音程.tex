\documentclass[a4paper,20pt]{article}
\usepackage{geometry}
\usepackage{savesym}
\usepackage{amsmath}
\usepackage{esint}
\usepackage{mathrsfs}
\usepackage{amssymb}

\usepackage[UTF8]{ctex}
\savesymbol{breve}
\usepackage{musixtex}
\restoresymbol{breve}{breve}

\input{musixlit}


\geometry{left=1.5cm, right=1.5cm, top=1.5cm, bottom=1.5cm}
\setlength{\lineskip}{0.75em}
\setlength{\parskip}{0.75em}

\begin{document}
\begin{center} 
 \Large \textbf{音乐理论基础}\par
 \textbf{十.音程}
\end{center}

\large 
\begin{center}
 \textbf{1.音程}\\
\end{center}

[\textbf{音程转位}] 音程的根音和冠音相互颠倒.\par
\qquad 音程转位可以在一个八度内进行, 也可以超过八度. 音程转位时, 可以移动根音, 也可以移动冠音, 也可以根音和冠音一齐移动.\par

[\textbf{音程转位时的规律}] \par
\qquad 1. 可以颠倒的音程总和是9. 当知道某一音程转位后的音程后, 便可以从9中减去原来音程的级数. 如七度(7)转位后(9-7=2)为二度;\par
\qquad 2. 除了纯音程外, 其他音程转位后都成为相反的音程.\par
\qquad \qquad 纯音程转位后为纯音程;\par
\qquad \qquad 大音程转位后为小音程, 小音程转位后为大音程;\par
\qquad \qquad 增音程转位后为减音程, 减音程转位后为增音程;\par
\qquad \qquad 倍增音程转位后为倍减音程, 倍减音程转位后为倍增音程.\par
\qquad 注意: 增八度转位后不是减一度, 而是减八度.\par

[\textbf{狭小音程构成法}] 由大小二度开始逐步构建音程的方法.\par
\qquad 小三度: 大二度和小二度结合; \qquad 大三度: 两个大二度结合;\par
\qquad 纯四度: 大二度和小三度, 或小二度和大三度结合;\qquad 增四度: 大二度和大三度结合;\par
\qquad 减五度: 两个小三度结合;\qquad 纯五度: 大三度和小三度结合;\par
\qquad 小六度: 纯四度和小三度结合;\qquad 大六度: 纯四度和大三度结合;\par
\qquad 小七度: 纯五度和小三度结合;\qquad 大七度: 纯五度和大三度结合.\par

[\textbf{构成和识别音程的方法}]\par
\qquad 1. 根据音程和级数构成和识别音程. 必须熟练地掌握音程包含的音数, 实际中应用不多;\par
\qquad 2. 根据狭小音程构成法来构成和识别音程. 方法虽然比较麻烦, 但在视听练耳中利用已掌握的狭小音程来构成广音程效果较好;\par
\qquad 3. 将音程与基本音级间的音程作比较. 方法比较简单, 但必须熟记基本音级间所形成的各种音程;\par
\qquad 4. 利用音程转位.\par

\clearpage

[\textbf{等音程}] 两个音程孤立起来听时, 具有同样的声音效果, 但在乐曲中的意义和写法不同. 等音程有两类.\par
\qquad 等音程是由于等音变化产生的, 主要在转调中使用.\par
\qquad 1. 音程中的两个音不因为等音变化而更改音程的级数;\par
\qquad 2. 由于等音变化而更改音程的级数.\par
\qquad 除三整音外, 每个自然音程(最简单且使用最多的音程)的等音程是某种增减音程.\par

[\textbf{协和音程}] 听起来悦耳, 融合的音程.\par
\qquad 1. 极完全协和音程: 声音完全合一的纯一度和几乎完全合一的纯八度;\par
\qquad 2. 完全协和音程: 声音相当融合的纯五度和纯四度;\par
\qquad 3. 不完全协和音程: 声音不很融合的大小三度和大小六度.\par

[\textbf{协和音程的特点}] 极完全协和音程和完全协和音程的声音会给人空的感觉, 不完全协和音程的声音较为丰满.\par

[\textbf{不协和音程}] 听起来比较刺耳, 彼此不很融合的音程.\par
\qquad 小二度, 大小七度, 所有的增减音程(包括增四度和减五度), 所有的倍增倍减音程.\par

[\textbf{音程转位对协和性的影响}] 协和音程转位后仍然是协和音程, 不协和音程转位后仍然是不协和音程.\par

[\textbf{稳定音程}] 由稳定音级构成的音程.\par
\qquad 1. 五声调式体系: 由于和声处理不同, 稳定音级也可能不同;\par
\qquad 2. 大小调体系: 在主音, 中音和属音之间构成的音程为稳定音程.\par

[\textbf{不稳定音程}] 由至少一个不稳定音级构成的音程.\par

[\textbf{不稳定音程的解决}] 解决不稳定音程的不稳定. \par
\qquad 最简单的方法是使不稳定音进行到最近的稳定音, 但解决时要避免平行五度和平行八度的进行.\par

\clearpage

\begin{center}
 \textbf{2.调式中的音程}\\
\end{center}

[\textbf{五声调式体系中的音程}] \par
\qquad 1. 五声调式: 只有纯一度, 大二度, 小三度, 纯四度以及它们的转位音程;\par
\qquad 2. 六声调式: 由清角和变宫引入了小二度;\par
\qquad \qquad 包含了纯一度, 大小二度, 大小三度, 纯四度以及它们的转位音程;\par
\qquad 3. 七声调式: 由于清乐音阶中清角和变宫(雅乐音阶中为变徵和变宫, 燕乐音阶中为清角和闰)的同时加入, 引入了增四度和减五度;\par
\qquad \qquad 包含了纯一度, 大小二度, 大小三度, 纯四度, 增四度以及它们的转位音程.\par

[\textbf{大小调体系中的音程}] \par
\qquad 1. 自然大小调: 与清乐音阶的宫调式及羽调式完全一样;\par
\qquad 2. 和声大小调: 相对自然大小调增加了增二度, 减七度, 增五度, 减四度. 这些音程为和声调式所特有, 故又叫做和声调式的特性音程;\par

[\textbf{不协和音程的解决}] 将不协和音程进行到协和音程.\par
\qquad 1. 基本方法: 不稳定音按照其倾向进行到最近的稳定音, 稳定音保持在原位不动或进行到其他稳定音;\par
\qquad 2. 对于增减音程: 在基本方法的基础上, 增音程解决时要使音程扩大, 减音程解决时要使音程缩小.\par
\qquad 不协和音程的解决在不同调式体系中有共同的规律, 但根据旋律不同又有所区别. 因此, 必须明确该音程所属的调性, 才能确定如何解决.\par

\begin{center}
 \textbf{3.音程在音乐中的应用及表现特性}\\
\end{center}

[\textbf{音程的形式}] 音程按照两个音级间的时间顺序有旋律形式和和声形式, 两种形式在音乐表现中是不相同的.\par
\qquad 旋律形式: 旋律进行的方向对旋律音程的表现有重大意义;\par
\qquad 和声形式: 音程的协和性和稳定性对和声音程的表现力有重大意义.\par

[\textbf{音程分类}] 根据两音之间的距离, 音程可以分为狭音程和广音程.\par
\qquad 狭音程: 同度, 二度, 三度;\par
\qquad 广音程: 四度, 五度, 六度, 七度, 八度等.\par

\clearpage

[\textbf{旋律音程}] 旋律狭音程一般具有平和, 安静的感觉;旋律广音程一般具有开阔, 跳跃的情感.\par
\qquad 旋律中应用最多的是狭音程, 广音程使用较少.\par
\qquad 1. 旋律一度: 作为同音重复的旋律发展因素之一;\par
\qquad 2. 旋律小二度: 是旋律进行中最典型最剧烈的音程之一;\par
\qquad 3. 旋律大二度: 是自然调式体系中旋律流畅进行的基本音程, 特别是在五声音阶调式中;\par
\qquad 4. 旋律三度: 在五声调式体系和大小调体系中有不同的表现特性. 在旋律进行中, 旋律三度有时服从于三和弦的整个表现力, 而失去它们独立的表现特性;\par
\qquad \qquad a. 五声调式体系: \par
\qquad \qquad \qquad 旋律小三度与旋律大二度同样是旋律流畅进行的基本音程, 这里小三度音程是级进而不是跳进;\par
\qquad \qquad \qquad 旋律大三度在五声调式中应用较少, 偶尔出现时也不常表现出大小调体系中那样的大调调性的特性;\par
\qquad \qquad b. 大小调体系: 大三度永远表现出大调特性, 小三度则永远表现出小调特性. 大小三度相互对比中更能体现各自的特征;\par
\qquad 5. 旋律四度: 旋律四度在进行中有不同作用, 但主属关系总是十分清楚. 四度下面的音永远为属, 上面的音则具有主的性质;\par
\qquad \qquad 弱起上行的旋律四度: 有鲜明的号召性和战斗性;\par
\qquad \qquad 下行旋律四度: 雄伟, 沉着, 肯定, 有力;\par
\qquad \qquad 上行四度: 往往具有抒情的性质;\par
\qquad 6. 旋律五度: 具有开阔性, 起旋律的中心作用. 主, 属, 下属的关系就是这一中心作用的表现;\par
\qquad 7. 旋律六度: 往往是旋律的开始音调, 也是旋律的高潮音程之一;\par
\qquad \qquad 在五声调式中, 大三度的转位(小六度)是一种常见的音程, 但往往没有大跳的感觉, 倒有某些级进音程的特点;\par
\qquad 8. 旋律七度: \par
\qquad \qquad 旋律小七度: 具有旋律高峰音程的表现特征;\par
\qquad \qquad \qquad 在五声调式体系中, 和六度音程一样, 具有级进效果;\par
\qquad \qquad \qquad 在大小调体系中, 小七度往往表现出属七和弦中根音和七度音的特征;\par
\qquad \qquad 旋律大七度: 由于过于宽广而较少应用;\par
\qquad 9. 旋律八度: 和六度, 七度一样, 具有高峰音程的特征.\par

\clearpage

[\textbf{和声音程的协和性}] 和声音程有协和和不协和两种.\par
\qquad 协和音程: 具有协调, 平静, 柔美的感觉;\par
\qquad 不协和音程: 紧张, 尖锐, 矛盾, 不安的感觉.\par

[\textbf{和声音程}]\par
\qquad 1. 和声一度: 两个声部在一个音中的融合, 它的特点是严整协调一直, 是支声复调音乐的特征;\par
\qquad 2. 和声小二度: 非常不协和的音程, 构成该音程的两个音似乎永远无法融合在一起;\par
\qquad 3. 和声大二度: 相当不协和, 但用来模仿打击乐有着很好的效果;\par
\qquad 4. 和声三度: 和旋律三度一样, 在大小调体系中最能说明大小调的特征, 并成为大小调体系中构成二部歌曲的基础;\par
\qquad 5. 和声纯四度: 在以五声音阶为基础的调式体系中, 起着重大的作用;\par
\qquad 6. 和声五度: 在大小调体系中表现出一种空的特性, 因此在和声学中平行五度是被禁止使用的. 但在五声调式中有时有着很好的效果;\par
\qquad 7. 和声六度: 在某些方面类似三度的表现力, 但不像大小三度那样具有鲜明的表现出大小调的调性. 和声六度有着独有的不确定性;\par
\qquad 8. 和声七度: 与和声二度相近, 不协和性是它最显著的特点, 往往是某一七和弦的化身;\par
\qquad 9. 和声八度: 与和声一度相近, 完全融合性是它的最大特点.\par

[\textbf{自然音程中三整音的特点}] 自然音程中的三整音(增四度和减五度)不论旋律形式还是和声形式, 其音响是紧张且不寻常的, 往往用来表现异常尖锐的或富于戏剧性的表情瞬间.\par

[\textbf{变化音程的特点}] 变化音程一般较少使用, 其特点是急剧要求解决, 并具有不协和性.\par

\end{document}
