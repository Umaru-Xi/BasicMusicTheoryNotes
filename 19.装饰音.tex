\documentclass[a4paper,20pt]{article}
\usepackage{geometry}
\usepackage{savesym}
\usepackage{amsmath}
\usepackage{esint}
\usepackage{mathrsfs}
\usepackage{amssymb}

\usepackage[UTF8]{ctex}
\savesymbol{breve}
\usepackage{musixtex}
\restoresymbol{breve}{breve}

\input{musixlit}


\geometry{left=1.5cm, right=1.5cm, top=1.5cm, bottom=1.5cm}
\setlength{\lineskip}{0.75em}
\setlength{\parskip}{0.75em}

\begin{document}
\begin{center} 
 \Large \textbf{音乐理论基础}\par
 \textbf{十九.装饰音}
\end{center}

\large 
\begin{center}
 \textbf{1.装饰音}\\
\end{center}

[\textbf{装饰音}] 用来装饰旋律的小音符及某些旋律型的特别记号.\par
\qquad 装饰音大部分由时值较短的辅助音(和主要音相差二度的音)构成, 演奏时它们的时值算在被装饰的音的时值之内, 或算在它们前面的音符的时值之内. 在记谱上它们不占基本拍子总时值的时间.\par

\begin{center}
 \textbf{2.倚音}\\
\end{center}

[\textbf{倚音分类}] 倚音分为短倚音和长倚音.\par

[\textbf{长倚音}] 由一个音组成, 在主要音的前面, 和主要音相距二度.长倚音一般带有强声.\par
\qquad 长倚音的演奏时值永远算在主要音内:\par
\qquad \qquad 如果主要音是单纯音符, 则长倚音占主要音符$\frac{1}{2}$的长度;\par
\qquad \qquad 如果主要音是附点音符, 则长倚音占主要音的$\frac{2}{3}$.\par
\qquad 长译音的标记: 不带斜线(斜线类似于$\hbar$)的不大于四分音符的小音符, 符干向上.长倚音的标记法在一般多年前被废除, 但许多古典乐曲中还经常遇到.\par
\qquad (未在MusixTex中找到对应代码编写方法, 需要查阅Kiwix)\par
\startextract 
\zendextract

[\textbf{短倚音}] 由一个或数个音组成, 这些音和主要音可以是级进或跳进关系, 可以在主要音前后.\par
\qquad 短倚音演奏的时值短暂, 并不带有强声.\par
\qquad 短倚音的标记: 用单符干记谱时, 符干永远朝上. \par
\qquad \qquad 如果是一个音, 则用带斜线的小八分音符标记;\par
\qquad \qquad 如果不止一个音, 则用组合起来的小的十六分音符标记.\par
\qquad (未在MusixTex中找到对应代码编写方法, 需要查阅Kiwix)\par
\startextract 
\zendextract

\clearpage

\begin{center}
 \textbf{3.波音}\\
\end{center}

[\textbf{波音}] 在两个主要音之间, 加入其上方或下方的短的辅助音而成. 波音有顺波音, 逆波音, 单波音和复波音.\par
\qquad 波音在演奏时一般占主要音的时间, 波音记号记在主要音的上方. 波音记号的上方或下方还可以带有变音记号, 用来表示辅助音的升高或降低.\par
\startextract
\Notes \shake n\qa e \Shake n\qa j \en\bar
\Notes \mordent n\qa l \Mordent n\qa g\en
\zendextract

[\textbf{顺波音的表记}] \par
\qquad 1. 小音符;\par
\qquad 2. 顺波音记号\quad {\shake e} \quad (复顺波音记号\quad {\Shake e}\quad ).\par

[\textbf{逆波音的表记}]\par
\qquad 1. 小音符;\par
\qquad 2. 逆波音记号\quad {\mordent e}\quad (复逆波音记号\quad {\Mordent e}\quad).\par

\begin{center}
 \textbf{4.回音}\\
\end{center}

[\textbf{回音}] 由四个或五个音组成的旋律型. 回音有顺回音和逆回音两种. \par
\qquad 由四个音组成的顺回音是由上方助音开始到主要音, 再到下方助音和主要音;\par
\qquad 由五个音组成的顺回音是由主要音开始, 后面与四个音的顺回音相同;\par
\qquad 逆回音和顺回音的方向相反.\par
\qquad 回音的奏法异常复杂且不固定, 在现代记谱法中早已不用.\par
\startextract
\Notes \turn n\qa e\en\bar
\Notes \backturn n\qa l\en
\zendextract

[\textbf{回音的表记}] 可用小音符或回音记号来表示. 顺回音记号\quad {\turn e}\quad , 逆回音记号可以是顺回音记号中间加一竖线或\quad {\backturn e}\quad. 回音记号可以记在音符上, 也可以记在两个音符之间. 回音记号上方和下方可以加变音记号, 用来表示助音的升高或降低.\par

\clearpage

\begin{center}
 \textbf{5.颤音}\\
\end{center}

[\textbf{颤音}] 颤音由主要音和它上方助音快速而均匀地交替而形成.\par

[\textbf{颤音的表记}] 用记号\quad {\tr e}\quad 或\quad {\tr e}\quad 加波浪线表示. 颤音记号记录在音符上方. 颤音记号上方可能有变音记号. 它是属于上面的助音的, 表示助音的升高或降低.\par

[\textbf{颤音的演奏}] 具体演奏方法可以参考古老乐曲的注释或音乐词典.\par
\qquad 1. 由主要音开始;\par
\qquad 2. 由上方助音开始;\par
\qquad 3. 由下方助音开始.\par

\end{document}
